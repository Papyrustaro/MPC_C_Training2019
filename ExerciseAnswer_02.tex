%------------------------------------- ページサイズなどの書式設定
%¥documentclass[a4j,twocolumn, dvipdfmx]{jsarticle} % 二段組の構成にする
%¥documentclass[a4j,notitlepage]{jsarticle} % タイトルだけのページを作らない
\documentclass[a4j,titlepage,dvipdfmx]{jsarticle}   % タイトルだけのページを作る
%------------------------------------- パッケージ読み込み
\newcommand{\stypath}{./sty}
\newcommand{\codepath}{./code/02_Answer}

\usepackage[ipaex]{pxchfon}
%\usepackage{itembkbx}
\usepackage{\stypath/listings}
\usepackage{ascmac}
\usepackage{\stypath/jlisting}
\lstset{% 
showstringspaces=false,%空白文字削除
language={C},% %言語選択
basicstyle={\upshape},% %標準の書体
identifierstyle={\small},% %キーワードでない文字の書体
ndkeywordstyle={\small},% %キーワードその2の書体
stringstyle={\small\ttfamily},% %””で囲まれた文字などの書体
frame={tb},% %枠、デザインなど
breaklines=true,% %行が長くなった時の自動改行
columns=[l]{fullflexible},% %書体による列幅の違いを調整するか
numbers=left,% %行番号を表示するか
xrightmargin=0zw,% %余白の調整?
xleftmargin=0zw,% %余白の調整
numberstyle={\scriptsize},%行番号の書体
stepnumber=1,% %行番号をいくつ飛ばしで表示するか
numbersep=1zw,% %行番号と本文の間隔
morecomment=[l]{//}% 
} 
\title{C言語講座第2回演習問題回答}
\author{那由多・榊}
\date{2019年5月16日}

\begin{document}

\section{榊くんからの挑戦状}
\subsection{問1}
\subsubsection{模範解答}
\lstinputlisting{\codepath/sakaki_exe01.c}
\subsubsection{解説}
%ここに解説

\subsection{問2}
\subsubsection{模範解答}
\lstinputlisting{\codepath/sakaki_exe02.c}
\subsubsection{解説}
%ここに解説

\subsection{問3}
\subsubsection{模範解答}
\lstinputlisting{\codepath/sakaki_exe03.c}
\subsubsection{解説}
%ここに解説

\subsection{問4}
\subsubsection{模範解答}
\lstinputlisting{\codepath/sakaki_exe04.c}
\subsubsection{解説}
%ここに解説

\end{document}