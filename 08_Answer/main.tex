%------------------------------------- ページサイズなどの書式設定
%¥documentclass[a4j,twocolumn, dvipdfmx]{jsarticle} % 二段組の構成にする
%¥documentclass[a4j,notitlepage]{jsarticle} % タイトルだけのページを作らない
\documentclass[a4j,titlepage,dvipdfmx]{jsarticle}   % タイトルだけのページを作る
%-------------------------------------コマンド定義
%styファイルのパスの簡略化
\newcommand{\stypath}{./sty}
%コードファイルの簡略化(./code/04のように毎回変更する)
\newcommand{\codepath}{./code}
%記事ファイルの簡略化(codepathと同様)
\newcommand{\articlepath}{./article}
%------------------------------------- パッケージ読み込み
\usepackage[ipaex]{pxchfon}
%\usepackage{itembkbx}
\usepackage{\stypath/listings}
\usepackage{ascmac}
\usepackage{\stypath/jlisting}
\lstset{%
showstringspaces=false,%空白文字削除
language={C},% %言語選択
basicstyle={\upshape},% %標準の書体
identifierstyle={\small},% %キーワードでない文字の書体
ndkeywordstyle={\small},% %キーワードその2の書体
stringstyle={\small\ttfamily},% %””で囲まれた文字などの書体
frame={tb},% %枠、デザインなど
breaklines=true,% %行が長くなった時の自動改行
columns=[l]{fullflexible},% %書体による列幅の違いを調整するか
numbers=left,% %行番号を表示するか
xrightmargin=0zw,% %余白の調整?
xleftmargin=0zw,% %余白の調整
numberstyle={\scriptsize},%行番号の書体
stepnumber=1,% %行番号をいくつ飛ばしで表示するか
numbersep=1zw,% %行番号と本文の間隔
morecomment=[l]{//}%
}

\title{C言語講座第8回解答}%何回か書き直す
\author{MPC部員}
\date{2019年7月11日}%日付も書き直す
\begin{document}
\maketitle
\section{演習問題(さんたろー作)}
\subsection{問1}
\subsubsection{解答例}
\lstinputlisting{\codepath/papyrus/exercise1.c}
\subsubsection{解説}
\begin{verbatim}
プログラム、テキストファイルの通り
\end{verbatim}

\section{演習問題}
\subsection{問1}
\subsubsection{問題}
\lstinputlisting{\codepath/sakaki/05_Answer1.cpp}
\subsubsection{解説}
出力をするプログラムを書くだけである。
出力をしないとプログラムでの処理が正しく実行されているかがわからない。
どのようなプログラムであれ使うので覚えておいて欲しい。

\section{演習問題}
\subsection{問2}
\subsubsection{問題}
\lstinputlisting{\codepath/sakaki/05_Answer2.cpp}
\subsubsection{解説}

四則演算をするプログラムである。
double answer=5*(3-1)+6.0/4;
printf("%lf",answer);
としても良いし、
printf("%lf",5*(3-1)+6.0/4);
としてもよい。
ここで、6.0としたのは、小数が出てくるためである。
もし6/4とした場合は答えが整数になってしまうはずだ。
わからなければ先輩に聞いて欲しい。
自己学習用検索ワード:「C言語 キャスト」

\section{演習問題}
\subsection{問3}
\subsubsection{問題}
\lstinputlisting{\codepath/sakaki/05_Answer3.cpp}
\subsubsection{解説}

前問同様、四則演算の問題である。
double型で入力を受け取り、和と積を出力すれば良い。
詳しい解説は前問と同様である。

\section{演習問題}
\subsection{問4}
\subsubsection{問題}
\lstinputlisting{\codepath/sakaki/05_Answer4.cpp}
\subsubsection{解説}

第1回演習と同じ問題である。
公式は((上底+下底)*高さ)/2であり、代入した値を用いてこの公式を当てはめれば良い。double型で受け取れば問2のようなことを考えなくて済む。


\section{演習問題}
\subsection{問5}
\subsubsection{問題}
\lstinputlisting{\codepath/sakaki/05_Answer5.cpp}
\subsubsection{解説}

if,else if文によって、1から12までの数字で出力内容を変える問題である。
今回は4通りなので、様々な考え方でコードが書ける。
解説のコードは、数値の範囲で場合分けをするので、普遍的に使える。
if(x==2||x==3||x==4)
のようなコードは、数が増えると書くのが面倒なので、非推奨である。

\section{演習問題}
\subsection{問6}
\subsubsection{問題}
\lstinputlisting{\codepath/sakaki/05_Answer6.cpp}
\subsubsection{解説}

for文によって29回繰り返すプログラムを書けばよい。
29回printf()を書いても良いが、例えば2000回出力したいとなった場合に変更が容易なのが、for文の強みである。

\section{演習問題}
\subsection{問7}
\subsubsection{問題}
\lstinputlisting{\codepath/sakaki/05_Answer7.cpp}
\subsubsection{解説}

7つの要素を持つ1次元配列に、for文で要素を入れ、2倍にしてから出力するプログラムを書く問題である。出力結果を同じにするだけなら配列は要らないが、そういう演習である。

\section{演習問題}
\subsection{問8}
\subsubsection{問題}
\lstinputlisting{\codepath/sakaki/05_Answer8.cpp}
\subsubsection{解説}

最大値,最小値共に似たようなコードで求めることが出来る。
最大値の場合も最小値の場合も、まず最初の要素を暫定の最大値、最小値とする。その後は

最大値:for文,if文によって、暫定の最大値より大きい要素を見つけた場合は暫定の最大値を更新する。
最小値:for文,if文によって、暫定の最小値より小さい要素を見つけた場合は暫定の最小値を更新する。

を要素の配列の最後まで行うことで、全体の最小値、最大値が求まる。
少し難しい考え方なので、わからなければ先輩に聞こう。


\section{演習問題}
\subsection{問9}
\subsubsection{問題}
\lstinputlisting{\codepath/sakaki/05_Answer9.cpp}
\subsubsection{解説}

単位行列:i=jを満たす値が1,他が0の行列
2重for文でi=jならば1,i!=jならば0と出力すれば良い。
各行の数字の出力が終わったら改行すれば綺麗な行列が表示される。

\section{演習問題}
\subsection{問10}
\subsubsection{問題}
\lstinputlisting{\codepath/sakaki/05_Answer10.cpp}
\subsubsection{解説}

前問での知識が活かせる問題である。入力された値の2重配列において、i=jを満たす部分のみ、和を求めればよい。

余力があれば、2次元配列を使わないでこの問題を解いてみると良い。
つまり、要素数N*Nの1次元配列で解いてみる。

\section{演習問題}
\subsection{問11}
\subsubsection{問題}
\lstinputlisting{\codepath/sakaki/05_Answer11.cpp}
\subsubsection{解説}

ここでは、簡単なソートであるバブルソートを書いてみた。
詳しく解説しようとすると、数ページは使うので、知りたければ下記のように〇〇ソートと検索して調べてみて欲しい。
自己学習用検索ワード:「バブルソート」「選択ソート」「挿入ソート」etc...
これらは学部2年以降ならば授業で習っているので、わからなければ先輩に聞くこと。

\section{演習問題}
\subsection{問12}
\subsubsection{問題}
\lstinputlisting{\codepath/sakaki/05_Answer12.cpp}
\subsubsection{解説}

不必要な文章によって、必要な情報が見づらくなっている。
しかし、現実で起きている問題を考える時はよくある話である。
要約すると、下記のようになる。
・5500に対して、1.3倍することを繰り返し、70億を超えるには何回1.3倍すればよだろうか。
つまり、for文で5500*1.3をして、70億を超えた時の回数を出力するだけである。

\section{演習問題}
\subsection{問13}
\subsubsection{問題}
\lstinputlisting{\codepath/sakaki/05_Answer13.cpp}
\subsubsection{解説}

まず、愚直に考えます。すると、この問題は下記のような問題であるとわかります。
・整数Xを、その各桁が等しくなるまで+1し、その回数を出力せよ。
それは即ち、X=111,222,333,444,555,666,777,888,999のいずれかになるまで増やし、その回数が出力されることを意味します。
このことから、この問題は下記のように言い換えることが出来ます。
・i=111からiを111ずつインクリメントするとき、Xが初めてi以下になる時のi-Xを出力する。

よって、解答のコードのように書けます。

\section{演習問題}
\subsection{問14}
\subsubsection{問題}
\lstinputlisting{\codepath/sakaki/05_Answer14.cpp}
\subsubsection{解説}

for文などにより、10個の整数を2で割っていきます。その時、10個目まですべてが偶数ならば+1し、割れなくなるまで繰り返します。この繰り返しの回数が答えです。


\section{演習問題}
\subsection{問15}
\subsubsection{問題}
\lstinputlisting{\codepath/sakaki/05_Answer15.cpp}
\subsubsection{解説}

まず、3N人の戦闘力を昇順にソートします。降順でも良いが、今回は昇順でソートしたと仮定して解説する。
グループの戦闘力が最も高いのは、昇順ソートにおいて、3N番目、3N-1番目と?番目がグループになった時である。
同様に、グループの戦闘力が最も低くなるのは、1番目、2番目、?番目がグループになった時である。
よって、ソート済みの配列において、
最大値は配列の3N-1番目
最小値は配列の2番目となります。
これを出力するだけです。

\end{document}
