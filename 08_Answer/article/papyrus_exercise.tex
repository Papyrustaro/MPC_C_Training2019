\section{演習問題(さんたろー作)}
\subsection{問1}
\subsubsection{解答例}
\lstinputlisting{\codepath/papyrus/exercise1.c}
\subsubsection{解説}
\begin{verbatim}
アスタリスクでの表現わかりずらいと思ったそこのあなた。
そうですよね。こんな書き方しなくていいです。

なぜ引数の配列が先頭アドレスだけしか渡さないはずなのに、
要素数を書いているの?という質問が聞こえて...くるといいな。
先頭アドレスだけだと、この配列が一体いくつの要素でひとつの行なのかわからないんですよね。
だから、一行三列だよーと教える必要があるのです。

引数を変えた、もっとわかりやすい書き方はこちら↓
https://qiita.com/Hiraku/items/babed27bc1d750c2e12d
\end{verbatim}

\subsection{問2}
\subsubsection{解答例}
\lstinputlisting{\codepath/papyrus/exercise2.c}
\subsubsection{解説}
\begin{verbatim}
文字を数値で引くと、文字コードが引かれる。
それだけ。

\end{verbatim}

\subsection{問3}
\subsubsection{解答例}
\lstinputlisting{\codepath/papyrus/exercise3.c}
\subsubsection{解説}
\begin{verbatim}
文字コードの勉強がてらやってもらいましたが、
ヘッダファイルで用意されています。↓
http://www.c-tipsref.com/reference/ctype/tolower.html
http://www.c-tipsref.com/reference/ctype/toupper.html
\end{verbatim}

\subsection{問4}
\subsubsection{解答例}
\lstinputlisting{\codepath/papyrus/exercise4.c}
\subsubsection{解説}
\begin{verbatim}
string.hなどの記事をあさってみると面白いと思います。
http://www.c-tipsref.com/reference/string.html
\end{verbatim}

\subsection{問5}
\subsubsection{解答例}
\lstinputlisting{\codepath/papyrus/exercise5.c}
\subsubsection{解説}
\begin{verbatim}
scanf関数が、ダメダメだという話。
私がこの記事を書いたのはだいぶ前で、今見ると修正点がちらほら(笑)。
まあ、Cは文字列の扱い苦手なんですよね...。
自分なりの入力関数をつくってみてはいかがでしょうか。

\end{verbatim}
