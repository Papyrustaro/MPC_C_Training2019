\section{演習問題}
\subsection{問1}
\subsubsection{解答例}
\lstinputlisting{\codepath/papyrus/exercise1.c}
\subsubsection{解説}
自分と敵の体力と攻撃力を変数に格納。while(1)で無限ループにし、if文でどちらかの体力が0になったらbreakによりループを脱出する。
自分の攻撃が先なので、計算式と条件式の順番は解答例の通り。

\subsection{問2}
\subsubsection{解答例}
\lstinputlisting{\codepath/papyrus/exercise2.c}
\subsubsection{解説}
\begin{verbatim}
int型で入力しているが、if文中の左辺の計算式の答えはdouble型になってることに注意。
int + double(今回は0.4,0.6)の答えはdouble。
問題文に書いてある通り、59.1以上なら60点になり合格。
\end{verbatim}

\subsection{問3}
\subsubsection{解答例}
\lstinputlisting{\codepath/papyrus/exercise3.c}
\subsubsection{解説}
種類関係なく個数を多く買いたいので、値段の小さいものから買っていく。
for分の判別式で所持金Nが対象の値段より多いかどうか、
対象の商品がまだあるかの2つを判別している。
この2つの条件を満たすときに、まだその商品買うことができる。\\
今回はfor文を利用してみたが、while文でも良さそうだ。
関数なども学ぶと、もっと完結に記述することができる。
