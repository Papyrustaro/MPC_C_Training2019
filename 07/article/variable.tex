\section{前置き}
今日はポインタについてやります。
このポインタというものは理解に困ったり、躓く人がかなり多いです。(情報系2年でもよくわからない人もいるかも)なぜそうなるかというと"あほほどややこしい"からです。

\subsection{変数について}
まず変数について、初めに変数について説明するとき、変数は数値などをメモリに名前を付けて保存すること、といったと思います。また、イメージしやすいように、変数は箱のようと言ったかもしれませんが、今回は前者の考え方が都合がよいです。

int型の変数は下の図のように4倍と分のメモリを占領します。
バイトとは0から255のいずれか1つの整数を表現できる単位のことです。詳しいことはいつかの授業でやると思います。気になる人は自分で調べてみるといずれ役立ちます。

\begin{table}[htb]
\begin{center}
\begin{tabular}{|c|c|}
\hline
0018FF4C & \\ \cline{1-1}
0018FF4D & num \\ \cline{1-1}
0018FF4E & \\ \cline{1-1}
0018FF4F & \\ \cline{1-1}
\hline

\end{tabular}
\caption{ int型の変数の格納例}
\end{center}
\end{table}

ちなみにこっちは配列の例です。上のメモリで下はアドレスとなっています。それぞれの単語についてはあとで説明するのでここでは図のみにしておきます。


\begin{table}[htb]
\begin{center}
\begin{tabular}{|c|c|c|c|c|c|c|c|}\hline

\multicolumn{4}{|c|}{a[0]}& \multicolumn{4}{|c|}{a[1]}\\ \hline
1&2&3&4&5&6&7&8\\ \hline

\end{tabular}
\caption{ int型の変数の格納例}
\end{center}
\end{table}