\section{榊作成 演習問題}
\begin{verbatim}
榊くんのぼやき

パピルスくんの問題でポインタの概要は理解出来ただろうか。
榊作の演習問題まで理解できれば、
ポインタや型毎の値の格納、配列の理解が常人を上回れるようになります。

答えは値が固定で出てくるか、変な値が出てくるかのいずれかである。(変な値にも意味はあるが。)
固定で出てくる場合はその数を、意味不明な値なら意味不明な値と予測してから実行してみること。
\end{verbatim}

\subsection{問1}
\begin{verbatim}
以下のソースコードを実行した時、出力結果はどのようになるか推測せよ。
答えは実行して確かめること。
もし実行しても過程がわからなければ先輩に聞き、理解すること。

#include<stdio.h>

int main(void){
	int a[5]={6,3,5,7,1};
	int *p=a;
	
	p+=4;
	(*p)++;
	
	printf("%d\n",(*--p)+5);
	
	return 0;
}
\end{verbatim}

\subsection{問2}
\begin{verbatim}
以下のソースコードを実行した時、出力結果はどのようになるか推測せよ。
答えは実行して確かめること。
もし実行しても過程がわからなければ先輩に聞き、理解すること。


#include<stdio.h>

int main(void){
	int a[5]={6,3,5,7,1};
	int *p=a+4;
	
	p=p-2;
	(*p)++;
	
	printf("%d\n",(*--p)+*(a+2));
	
	return 0;
}
\end{verbatim}

\subsection{問3}
\begin{verbatim}
以下のソースコードを実行した時、出力結果はどのようになるか推測せよ。
printfのフォーマット指定子が%dであることに注意し、答えは実行して確かめること。
もし実行しても過程がわからなければ先輩に聞き、理解すること。
サブ講師,受講者用参考文献
その1:https://www.cc.kyoto-su.ac.jp/~yamada/programming/float.html
その2:https://www.k-cube.co.jp/wakaba/server/floating_point.html


#include<stdio.h>

int main(void){
	double x[6]={ 0 , 0.3 , 1.3 , 5.1 , 1.0 , 2.0 };
	double *p=x+4;
	
	p=p-3;
	*p+=2;
	
	printf("%d\n",*p);
	printf("%d\n",*(p+1));
	
	return 0;
}
\end{verbatim}

\subsection{問4}
\begin{verbatim}
おまけ問題(問4):
この問題はポインタを使いません。前回習った関数を使う問題です。

以下に与えられるのは、ある数Xが素数かどうかを判定する関数prime_numberである。
素数ならば1,素数でなければ0を返り値としている。
この関数を用いて、1からNまでの各数字x_iに対し、
素数ならばprintf("%d is Prime number!!\n",i);を実行し、
素数でないならばprintf("Oh... %d is not Prime number...\n",i);を実行する関数を作成せよ。
なお、Nは10000以下である。

int prime_number(int x){
	int i;
	if(x==1){return 0;}
	
	for(i=2;i*i<=x;i++){
		if(x%i==0){return 0;}
	}
	
	return 1;
}

sample input
5

sample output
Oh... 1 is not Prime number...
2 is Prime number!!
3 is Prime number!!
Oh... 4 is not Prime number...
5 is Prime number!!



おまけのおまけ

この問題を解いてみると、全ての値について、
素数かどうかを判定するため、非常に無駄が多いことがわかります。
「エラトステネスの篩」というアルゴリズムを使うと、
この問題が例えばN=10^7とかでも、現実的な時間で解くことが出来ます。
興味があれば調べて実装してみたり、どこかのサイトからコピペして動きを確認してみましょう。
(ただし、出力には非常に時間がかかります。)
\end{verbatim}
