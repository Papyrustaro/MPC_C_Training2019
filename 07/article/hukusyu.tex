\section{前回の復習}
\subsection{関数}
関数の宣言は以下のように行います。また、関数に与える値を引数(ひきすう)、関数が返す値を返り値といいます。

\begin{itembox}{関数の宣言}
\begin{verbatim}
返り値の型 変数名(引数2、引数2, ... ){
関数の処理
}
\end{verbatim}
\end{itembox}

\subsection{関数呼び出し例}
作った関数は以下のように使います。
\lstinputlisting{\codepath/func1.c}
\begin{itembox}{実行結果}

数字を2つ入力してください\\
33 4\\
37
\end{itembox}
余裕があったら復習に値を変えたりscanfを使って値を自分で入れたりしてみてください。
\subsection{復習問題}
引数として円錐の半径と高さを取り、その体積を返す関数を作成せよ。
