\section{配列とポインタ}

\subsection{配列との関係性}
配列とポインタの深い関係について簡単に説明します。配列は先ほど図で示した様にメモリ上に連続した領域をとることが特徴です。(これが長所にも短所にもなりうる)
まずは以下のようなプログラムを実行してみましょう。

\lstinputlisting{\codepath/point3.c}

\begin{itembox}{出力結果}
3\\
2

\end{itembox}
これが配列の真相ですね。

ちなみにこんな感じでfor文を回すことによって、普段の配列と同様な使い方もできます。

\lstinputlisting{\codepath/point4.c}
\begin{itembox}{出力結果}

5 7 3 4 2 8\\
5 7 3 4 2 8
\end{itembox}

\subsection{関数のスコープとの関係性}
前回スコープで値の共有できる範囲を確認しました。下のプログラムは確認用です。写さなくてもいいので見てみてください。

\lstinputlisting{\codepath/point5.c}

\begin{itembox}{出力結果}
2つの整数を入力しよう\\
1 5\\
x = 1 y = 5

\end{itembox}
この結果を見てわかるようにfunc関数で入れ替えの操作を行っても入力した値は入れ替わりません。じゃーどーするのかというと

\lstinputlisting{\codepath/point6.c}

\begin{itembox}{出力結果}

2つの整数を入力しよう\\
1 5\\
x = 5 y = 1
\end{itembox}
このように関数内でポインタを渡しあうということをすると値がしっかりと入れ替わります。

\subsection{配列を引数にとりたい}
次は配列を引数にいれてみましょう。\\
まずはみなさんおなじみ(?)かもしれない、配列の各要素を2倍するプログラムです。

\lstinputlisting{\codepath/point7.c}


\begin{itembox}{出力結果}
10 14 6 8 4 16
\end{itembox}

下のプログラムは上のプログラムの配列の各要素を2倍にする部分を関数にしたものです。ちなみにプログラム内での仮引数がa*となっていますが1[]とかいてもかまいません。
試してみましょう。(どちらもアドレスを示しています)

\lstinputlisting{\codepath/point8.c}


\begin{itembox}{出力結果}
10 12 6 8 4 16
\end{itembox}s
