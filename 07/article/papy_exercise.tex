\section{演習問題(さんたろー作)}
\subsection{最初に}
\begin{verbatim}
ポインタの理解は一筋縄ではいかない。
また、関数や配列などの知識をこれから利用していくことになる。
慌てず、ひとつひとつ理解していこう。
(正直参照渡しさえ理解すれば最低限は使える...はず)

全体を通しての参考文献(ポインタ)
https://9cguide.appspot.com/15-01.html
\end{verbatim}
\subsection{問1(難易度☆)}
\subsubsection{問題文}
\begin{verbatim}
以下のプログラムを実行し、なぜこのような結果になるのか考えなさい。
\end{verbatim}

\subsubsection{コード}
\begin{verbatim}
#include <stdio.h>

int main(void){
    int n = 10;
    int *p = &n;

    printf("%d\n", *p);
    printf("%p\n", p);

    *p += 5;
    p += 5;

    printf("%d\n", n);
    printf("%p\n", &n);

    printf("%d\n", *p);
    printf("%p\n", p);
}
\end{verbatim}

\subsection{問2(難易度☆)}
\subsubsection{問題文}
\begin{verbatim}
以下のプログラムを実行し、なぜこのような結果になるのか考えなさい。
\end{verbatim}

\subsubsection{コード}
\begin{verbatim}
#include <stdio.h>

int main(void){
  int a[3] = {1, 2, 3};
  int *p = a;
  int *q = &a[0];

  printf("%p\n", p);
  printf("%p\n", q);

  printf("\n");
  for(int i = 0; i < 3; i++){
    printf("%d\n", *(p + i));
  }
  for(int i = 0; i < 3; i++){
    printf("%d\n", *(q + i));
  }
}

\end{verbatim}

\subsection{問3(難易度☆☆)}
\subsubsection{問題文}
\begin{verbatim}
引数のアドレスにあるint型の値を10倍にする関数を作り、実行しなさい。
10倍する整数はキーボードから入力できるように。
(オーバーフロー等は考えなくてよい)

返り値: void
引数: int *(int型ポインタ変数)
\end{verbatim}

\subsubsection{実行例}
\begin{verbatim}
n = 3
n = 30

n = -10
n = -100
\end{verbatim}

\subsection{問4(難易度☆☆)}
\subsubsection{問題文}
\begin{verbatim}
要素数5のint型の配列をmain関数で用意し、scanf関数等で入力。
その配列の値を先頭から順に表示しなさい。

ただし、配列の先頭アドレスを引数として渡し、
自作関数内で表示すること。
最初からわからない人は、問5のひな形を参考にすること。

返り値: void
引数: int[](要素数5の配列の先頭アドレス)
\end{verbatim}

\subsubsection{実行例}
\begin{verbatim}
a[0] = 1
a[1] = 2
a[2] = 3
a[3] = 4
a[4] = 5

a[0] = 1
a[1] = 2
a[2] = 3
a[3] = 4
a[4] = 5
\end{verbatim}

\subsection{問5(難易度☆☆☆)}
\subsubsection{問題文}
\begin{verbatim}
下のひな形に、int型の配列の要素の順番を、逆順にする関数を完成させ、
main関数内で呼び出しなさい。

返り値: void
引数: int *(int型配列の先頭アドレス)、配列の要素数


/*以下プログラムのひな形*/
#include <stdio.h>

//プロトタイプ宣言

int main(void){
  int n;
  printf("配列の要素数n = "); scanf("%d", &n);
  int a[n];

  for(int i = 0; i < n; i++){
    printf("a[%d] = ", i); scanf("%d", &a[i]);
  }
  //自作関数を実行

  printf("変換後\n");
  for(int i = 0; i < n; i++){
    printf("a[%d] = %d\n", i, a[i]);
  }
}

//自作関数
\end{verbatim}

\subsubsection{実行例}
\begin{verbatim}
配列の要素数n = 3
a[0] = 1
a[1] = 2
a[2] = 3
変換後
a[0] = 3
a[1] = 2
a[2] = 1


配列の要素数n = 4
a[0] = 1
a[1] = 2
a[2] = 3
a[3] = 100
変換後
a[0] = 100
a[1] = 3
a[2] = 2
a[3] = 1
\end{verbatim}

\subsection{最後に(その他参考url)}
\subsubsection{Cをブラウザ上で実行したい}
\begin{verbatim}
https://wandbox.org/
\end{verbatim}
\subsubsection{ubuntuを仮想で使ってみたい方}
\begin{verbatim}
https://qiita.com/Aruneko/items/c79810b0b015bebf30bb
https://qiita.com/ykawakami/items/4bae371932110b2e25e3
\end{verbatim}
\subsubsection{もっといろんな演習問題をやりたい方}
\begin{verbatim}
https://kenkoooo.com/atcoder/#/table//
過去のC講座資料(mpcのサイトから見れます)
\end{verbatim}
\subsubsection{その他独学サイト}
\begin{verbatim}
https://dotinstall.com/
https://paiza.jp/
https://prog-8.com/
(独学サイトは本当に最初だけでいい気がする)
\end{verbatim}
