\section{ポインタ}
\subsection{アドレスについて}
 ではそれぞれ本題のポインタに入ります。まずは、アドレスについて説明します。まずは以下のプログラムを実行してみましょう。

\lstinputlisting{\codepath/point1.c}

\begin{itembox}{出力結果}

hensu1の値 : 2\\
アドレス : 0061FF2C\\
hensu2の値 : 2 \\
アドレス : 0061FF28
\end{itembox}
実行結果のアドレスの値はたぶん同じになっていないと思います。アドレスの値が実際に変数の値が入っている場所です。

このプログラムは変数のアドレスを確認するということを行っています。また8行目に\&hensu1とありますが、この\&が変数のアドレスを表示できるようにしているものです。これによってメモリ上のどこかに変数があるかということが分かるようになっています。

今までscanfの引数で\&をつけないといけなかったのは変数のアドレスを指定して、そこに入力した値を渡していたからなんですね。
厳密には変数の値からアドレスを調べるようです。

\subsection{ポインタについて}
日常生活で、これを見たい、というときに指とかペンで刺したりしますよね。ポインタを使うことでコンピュータの、この変数の場所を教えてくれ、という指示にたいして、メモリ空間の何番目にあるよ、という感じで教えることができます。ポインタはそんなかんじです
厳密にはアドレスから変数の値を返すというものです。

\lstinputlisting{\codepath/point2.c}

\begin{itembox}{出力結果}
3
\end{itembox}







