%------------------------------------- ページサイズなどの書式設定
%¥documentclass[a4j,twocolumn, dvipdfmx]{jsarticle} % 二段組の構成にする
%¥documentclass[a4j,notitlepage]{jsarticle} % タイトルだけのページを作らない
\documentclass[a4j,titlepage,dvipdfmx]{jsarticle}   % タイトルだけのページを作る
%------------------------------------- パッケージ読み込み
\newcommand{\stypath}{./sty}
\newcommand{\codepath}{./code}
\newcommand{\articlepath}{./article}
\usepackage[ipaex]{pxchfon}
%\usepackage{itembkbx}
\usepackage{\stypath/listings}
\usepackage{ascmac}
\usepackage{\stypath/jlisting}
\lstset{% 
showstringspaces=false,%空白文字削除
language={C},% %言語選択
basicstyle={\upshape},% %標準の書体
identifierstyle={\small},% %キーワードでない文字の書体
ndkeywordstyle={\small},% %キーワードその2の書体
stringstyle={\small\ttfamily},% %””で囲まれた文字などの書体
frame={tb},% %枠、デザインなど
breaklines=true,% %行が長くなった時の自動改行
columns=[l]{fullflexible},% %書体による列幅の違いを調整するか
numbers=left,% %行番号を表示するか
xrightmargin=0zw,% %余白の調整?
xleftmargin=0zw,% %余白の調整
numberstyle={\scriptsize},%行番号の書体
stepnumber=1,% %行番号をいくつ飛ばしで表示するか
numbersep=1zw,% %行番号と本文の間隔
morecomment=[l]{//}% 
} 
\title{C言語講座第七回・ポインタ}
\author{MPC部員}
\date{2019年7月4日}
\begin{document}
\maketitle
\section{前回の復習}
\subsection{ポインタ}
アドレスは定数が保存されている場所の番号のようなものです。アドレスを参照する場合は変数か型に*をつけることででポインタ変数にする必要があります。
ポインタとはその場所を保存するものです。また、ポインタを有効に活用することで、
関数間での変数の値の変更ができたり、配列を扱うことができます。

\begin{itembox}{ポインタの宣言方法}
\begin{verbatim}
型名 *ポインタ名
///
型名* ポインタ名

例)int *num ;
   int* num,num2;
\end{verbatim}
\end{itembox}
このときの注意なのですが、例で示した下の方は2個目の変数はポインタ変数ではありません。実はnum2は通常の変数を取ります。つまり、numはポインタ変数で、num2はint型の変数となります。わかりづらいので基本、上の型宣言がいいと思います。
\begin{table}[htb]
\begin{center}
\begin{tabular}{|c|c|}
\hline
アドレス & 配列\\ \hline
0018FF4C & a[0]\\ \hline
0018FF4D & a[1]\\ \hline
0018FF4E & a[2]\\ \hline
0018FF4F & a[3]\\ \hline\hline\hline
0018FF60 & \&a[0]\\
\hline

\end{tabular}
\caption{ int型の変数の格納例}
\end{center}
\end{table}
この場合0018FF60には\&a[0]、つまりa[0]のアドレス0018FF4Cが入っています。
\subsection{復習問題}
\noindent
1.変数aを宣言し、そのアドレスを表示せよ\\
2.2つの変数の値を入れ替えるvoid型のswap関数を作成し,変数a,bを入れ替えよ.\\
分からない場合はどんどん前回資料や末尾の参考文献などを参照してみてください。


%復習
\section{前置き}
今日はポインタについてやります。
このポインタというものは理解に困ったり、躓く人がかなり多いです。(情報系2年でもよくわからない人もいるかも)なぜそうなるかというと"あほほどややこしい"からです。

\subsection{変数について}
まず変数について、初めに変数について説明するとき、変数は数値などをメモリに名前を付けて保存すること、といったと思います。また、イメージしやすいように、変数は箱のようと言ったかもしれませんが、今回は前者の考え方が都合がよいです。

int型の変数は下の図のように4倍と分のメモリを占領します。
バイトとは0から255のいずれか1つの整数を表現できる単位のことです。詳しいことはいつかの授業でやると思います。気になる人は自分で調べてみるといずれ役立ちます。

\begin{table}[htb]
\begin{center}
\begin{tabular}{|c|c|}
\hline
0018FF4C & \\ \cline{1-1}
0018FF4D & num \\ \cline{1-1}
0018FF4E & \\ \cline{1-1}
0018FF4F & \\ \cline{1-1}
\hline

\end{tabular}
\caption{ int型の変数の格納例}
\end{center}
\end{table}

ちなみにこっちは配列の例です。上のメモリで下はアドレスとなっています。それぞれの単語についてはあとで説明するのでここでは図のみにしておきます。


\begin{table}[htb]
\begin{center}
\begin{tabular}{|c|c|c|c|c|c|c|c|}\hline

\multicolumn{4}{|c|}{a[0]}& \multicolumn{4}{|c|}{a[1]}\\ \hline
1&2&3&4&5&6&7&8\\ \hline

\end{tabular}
\caption{ int型の変数の格納例}
\end{center}
\end{table}%変数の再確認
\section{ポインタ}
%変数とポインタ
\section{配列とポインタ}
%配列とポインタの関係
\section*{参考文献}
\noindent
[1]昨年までの講座資料\newline
[2]\href{http://9cguide.appspot.com}{苦しんで覚えるC言語}\newline
[3]\href{https://yukicoder.me/}{yukicoder}\newline
[4]\href{http://www.c-tipsref.com/reference/string.html}{C言語関数辞典}
\end{document}
