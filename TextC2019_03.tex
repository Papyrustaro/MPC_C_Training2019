%------------------------------------- ページサイズなどの書式設定
%¥documentclass[a4j,twocolumn, dvipdfmx]{jsarticle} % 二段組の構成にする
%¥documentclass[a4j,notitlepage]{jsarticle} % タイトルだけのページを作らない
\documentclass[a4j,titlepage,dvipdfmx]{jsarticle}   % タイトルだけのページを作る
%------------------------------------- パッケージ読み込み
\newcommand{\stypath}{./sty}
\newcommand{\codepath}{./code/03}
\usepackage[ipaex]{pxchfon}
%\usepackage{itembkbx}
\usepackage{\stypath/listings}
\usepackage{ascmac}
\usepackage{\stypath/jlisting}
\lstset{% 
showstringspaces=false,%空白文字削除
language={C},% %言語選択
basicstyle={\upshape},% %標準の書体
identifierstyle={\small},% %キーワードでない文字の書体
ndkeywordstyle={\small},% %キーワードその2の書体
stringstyle={\small\ttfamily},% %””で囲まれた文字などの書体
frame={tb},% %枠、デザインなど
breaklines=true,% %行が長くなった時の自動改行
columns=[l]{fullflexible},% %書体による列幅の違いを調整するか
numbers=left,% %行番号を表示するか
xrightmargin=0zw,% %余白の調整?
xleftmargin=0zw,% %余白の調整
numberstyle={\scriptsize},%行番号の書体
stepnumber=1,% %行番号をいくつ飛ばしで表示するか
numbersep=1zw,% %行番号と本文の間隔
morecomment=[l]{//}% 
} 
\title{C言語講座第二回}
\author{那由多(堀越亮我)}
\date{2019年5月16日}
\begin{document}
\maketitle
\section{前回の復習}
\begin{enumerate}
\item if,if-else if-else if-else
\item switch-case
\end{enumerate}
\section{値の代入操作とインクリメント・デクリメント}
\subsection{値の代入操作}
第一回の時にC言語における=は左の変数に右の数式の計算結果を代入するという話をしました.\\
それの復習を行います.
\begin{lstlisting}
number=334;
\end{lstlisting}
この代入操作には何も違和感がありませんが次の場合はどうでしょう.
\begin{lstlisting}
number=number+1;
\end{lstlisting}
この代入操作はおかしく見えますが正しい操作です.これはnumberに+1した数字をnumberに代入するという操作です.
この操作は少し省略して書くことができます.

\begin{lstlisting}
number+=1
\end{lstlisting}
この表記は先ほどと同じ操作です.
具体的には
\begin{itembox}
"変数" "計算操作"="変数を操作したい値"
\begin{itembox}
という風になります.
少し見辛いですがよく使う表現ですのでよく確認しておきましょう

\subsection{インクリメント・デクリメント}
%ここからはまだ
\subsection{前置と後置}
\section{繰り返し操作}
\section{制御}
\section{演習問題}
問1から問3までがfor文の動作確認\\
問4はfor文の中で条件分岐をする練習\\
榊くんの挑戦状は、難易度BASICから手ごわい問題ばかりだが、解ければ本講座の内容は完全に理解したといえるだろう。\\
EX問題は今までの講座の知識のみでは解くことがほぼ不可能なので、晩御飯奢られチャレンジをしたい人は頑張って予習してきてください。\\
ただし、今回のEX課題は難易度がおかしいので多分無理です。\\

\subsection{問1}
出題意図:for文によりある回数だけ同じことを繰り返せることを理解する。\\

Steel is my body, and fire is my blood.I have created over a thousand blades. Unknown to Death. Nor known to Life.Have withstood pain to create many weapons.Yet, those hands will never hold anything.So as I pray, unlimited blade works.\\
を10回出力せよ。1回毎に改行すること。なお、これはFateが元ネタである。\\

\subsection{問2}
出題意図:for文のループカウンタを使う練習をする。\\
\\
1から200までの総和を求めるプログラムを作成せよ。数学で習った公式は使わずに実装しなさい。\\
即ち、for文を用いて1から200までを全て足し算しなさい。\\

\subsection{問3}
出題意図:for文の初期条件、終了条件を大まかに理解する。\\
\\
問2のコードを改造し、XからY(例えば20から183)までの総和を求めるプログラムにしなさい。\\
ただし、X,Yは標準入力から受け取ること(つまりscanfで読み取ること)。\\

\subsection{問4}
出題意図:for文内でif文を使って制御する方法を身につける。\\
\\
問3のコードを改造し、3の倍数の時には加算しないようにせよ。\\


\section{榊くんの挑戦状}

\subsection{BASIC}
出題意図:2重for文とその動作を理解する。\\
\\
九九の表を出力せよ。なお、printfのフォーマット指定子を\%dから\%2dとすると出力がきれい並ぶようになる。\\
表らしくするため、各段の出力毎に改行すること。\\

\subsection{BASIC}
出題意図:典型問題「FizzBuzz」から,for文とif,if else文を理解する。\\
\\
for文をN回回す。1からNまでを出力せよ。\\
ただし3の倍数の時は「Fizz」を,5の倍数の時は「Buzz」を,3の倍数かつ5の倍数の時は「FizzBuzz」を出力しなさい。\\

\subsection{BASIC}
出題意図:問題文の求める意味を理解してコードを書こう。\\
\\
int型の変数aを宣言し、20回だけ標準入力から整数を受け取る。\\
この1回目、3回目、5回目...つまり奇数番目に読み取った整数のみを出力しなさい。\\
例を挙げる。\\
1 64 2 7 5 75 334 84 32 7 43 74 24 21 12 55 184 24 376 256\\
が与えらえた場合は,\\
1 2 5 334 32 43 24 12 184 376\\
と出力すること。空白区切りが難しければ、\\
1\\
2\\
5 \\
334 \\
32 \\
43 \\
24 \\
12 \\
184 \\
376\\
として出力毎に改行しても良い。\\

\section{榊くんの挑戦状EX}
今回のEX問題は2つの小課題に分類される。\\
EX課題問1かEX課題問2に正解すれば500円以内の品物を,両方を正解した人には晩御飯を奢ります。\\
(新学部1年先着1名)\\

\subsection{EX問1}
フィボナッチ数列というものがある。ここでは、その定義は以下のようなものである。\\
f[0]=0,f[1]=1,\\
f[n]=f[n-1]+f[n-2](n>=2)\\
まず、フィボナッチ数列の第0項から第80項までを順に出力せよ。\\
ただし、配列を使用してはいけない。\\
int型ではオーバーフローする点に注意すること。\\
\\
これが出来たら、EX問2に答えよ。以下のEX問2ではフィボナッチ数列を求める時に、配列を使用してよい。\\

\subsection{EX問2}
フィボナッチ数列の第x項から第y項(0<x<y<70)までに,項が素数となるものの総数を求めよ。\\
ただし、配列にフィボナッチ数列の項を予め埋め込むこと、それが素数であるかどうかを埋め込むことは禁ずる。\\
つまり、プログラム内でフィボナッチ数列を作り、素数であるかの判定もせよ。\\
x,yは10回与えられ、10回分の解をそれぞれ出力すること。ただし、遅くても2,3秒程度ですべての解を出力しなければ不正解とする。\\

\end{document}
