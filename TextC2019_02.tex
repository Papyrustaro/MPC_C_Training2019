%------------------------------------- ページサイズなどの書式設定
%¥documentclass[a4j,twocolumn, dvipdfmx]{jsarticle} % 二段組の構成にする
%¥documentclass[a4j,notitlepage]{jsarticle} % タイトルだけのページを作らない
\documentclass[a4j,titlepage,dvipdfmx]{jsarticle}   % タイトルだけのページを作る
%------------------------------------- パッケージ読み込み
\newcommand{\stypath}{./sty}

\usepackage[ipaex]{pxchfon}
%\usepackage{itembkbx}
\usepackage{\stypath/listings}
\usepackage{ascmac}
\usepackage{\stypath/jlisting}
\lstset{% 
showstringspaces=false,%空白文字削除
language={C},% %言語選択
basicstyle={\upshape},% %標準の書体
identifierstyle={\small},% %キーワードでない文字の書体
ndkeywordstyle={\small},% %キーワードその2の書体
stringstyle={\small\ttfamily},% %””で囲まれた文字などの書体
frame={tb},% %枠、デザインなど
breaklines=true,% %行が長くなった時の自動改行
columns=[l]{fullflexible},% %書体による列幅の違いを調整するか
numbers=left,% %行番号を表示するか
xrightmargin=0zw,% %余白の調整?
xleftmargin=0zw,% %余白の調整
numberstyle={\scriptsize},%行番号の書体
stepnumber=1,% %行番号をいくつ飛ばしで表示するか
numbersep=1zw,% %行番号と本文の間隔
morecomment=[l]{//}% 
} 
\title{C言語講座第二回・条件分岐}
\author{那由多(堀越亮我)}
\date{2019年5月16日}
\begin{document}
\maketitle
\section{前回の復習}
\begin{enumerate}
\item ubuntuの操作
\item 画面への出入力(printf,scanf)
\item 変数の宣言(int float double)
\item 四則演算(+ - * / ())
\end{enumerate}
\section{条件分岐}
\subsection{比較演算子}
条件分岐させるとき条件を設定する必要があります.\\
そのときある値と同値のときやそれより大きな時,小さな時にプログラムを実行すると言うような使い方をします.
\begin{table}[h]
\begin{tabular}{|c|c|}
\hline
比較演算子 & 意味 \\ \hline
a==b & 同値 \\ \hline
a\textless{}b & aよりbが大きい \\ \hline
a\textgreater{}b & aよりbが小さい \\ \hline
a\textless{}=b & bがa以上 \\ \hline
a\textgreater{}=b & bがa以下 \\ \hline
a!=b & aがbではない \\ \hline
\end{tabular}
\end{table}
\subsection{if文}
今回は条件分岐を行います.難しいように感じますが簡単に言うと"もし〜だったら・・・する"っていうだけです.\\
実際その名の通り関数はifを使います.それではコードを書いてみましょう
\begin{itembox}{if}
if(条件){
	プログラムの中身
}
\end{itembox}
\begin{lstlisting}
#include<stdio.h>
int main(void){
	int data;
    printf("iPhone9は存在しますか? 0:はい\t 1:いいえ\n");
    if(data==0){
    	printf("正解です");
    }
    return 0;
}
\end{lstlisting}
\begin{itembox}{実行結果}
iPhone9は存在しますか? 0:はい      1:いいえ
0
正解です

iPhone9は存在しますか? 0:はい      1:いいえ
1
\end{itembox}
これによって特定の条件のときif文の中身が実行されます.
\subsection{else}
elseを使う事によってifで設定した条件とは反する場合に実行するプログラムを書くことができます.\\
先ほどのプログラムをelseを使って書き換えてみましょう
\begin{itembox}{else}
if(条件) {
	プログラム
}
else{
	プログラム
}
\end{itembox}
\begin{lstlisting}
#include<stdio.h>
int main(void){
	int data;
    printf("iPhone9は存在しますか? 0:はい\t 1:いいえ\n");
    if(data==0){
    	printf("正解です");
    }
    else{
    	printf("不正解です");
    }
    return 0;
}
\end{lstlisting}
\begin{itembox}{実行結果}
iPhone9は存在しますか? 0:はい\t 1:いいえ
0
正解です

iPhone9は存在しますか? 0:はい\t 1:いいえ
1
不正解です.
\end{itembox}
このような文章を作ることができます.\\
\subsection{else if}
else ifを使う事によっていくつもの条件を設定することができます.\\
\begin{itembox}{else if}
if(条件){
	プログラムの中身
}
else if(条件){
	プログラムの中身
}
else{
	プログラムの中身
}
\end{itembox}
\begin{lstlisting}
#include<stdio.h>
int main(void){
    int data;
    printf("1~3の好きな数字を入力してください\n");
    scanf("%d",&data);
    if(data==1){
    	printf("今日はSAINOに行こう\n");
    }
    else if(data==2){
    	printf("今日はロブジェに行こう\n");
    }
    else if(data==3){
    	printf("今日はチャイナに行こう\n");
    }
    else{
    	printf("1~3の数字を入力してください\n");
    }
    return 0;
}
\end{lstlisting}

\begin{itembox}{実行結果}
1~3の好きな数字を入力してください
1
今日はSAINOに行こう

1~3の好きな数字を入力してください
2
今日はロブジェに行こう

1~3の好きな数字を入力してください
3
今日はチャイナに行こう

1~3の好きな数字を入力してください
4
1~3の数字を入力してください
\end{itembox}

\subsection{複数条件の分岐}
\subsubsection{論理演算子}
条件の複数入力したい場合はif文を入れ子にするか次の表の論理演算子を使います.
\begin{table}[htb]
\begin{tabular}{|c|c|}
\hline
複数条件の意味 & \multicolumn{1}{l|}{論理演算子} \\ \hline
And     & \&\&                            \\ \hline
Or      & \verb+|+	\verb+|+                          \\ \hline
\end{tabular}
\end{table}
\lstinputlisting{./code/02/Logic.c}
\subsubsection{ネスト構造}
if文を入れ子構造にすることもできます.これをネスト構造という風に呼びます.
\lstinputlisting{./code/02/nest.c}
先ほどと同じプログラムを違う書き方をしてみました.このように一つのプログラムでも様々な書き方があります.\\
\subsection{switch文}
if文と同じような条件分岐としてswitch-case文というものがあります.\\
下のプログラムを実行してください\\
\lstinputlisting{./code/02/switch.c}
swich文はcaseごとにbreak文をつける約束があります.\\
外して実行してみましょう.\\
表示がおかしくなりますので気をつけましょう.\\
\section{おまけ}
\subsection{書き方}
C言語のプログラムは書き方にいろいろあります.例えばインデントの有無や中括弧の位置,そもそもほぼ改行をしないコードの書き方もあります.いくつか例をあげますが,僕自身の見解としてはせめて改行とインデントはしたほうが読みやすいかと思います.
\begin{lstlisting}
#include<stdio.h>
int main(void){int data;printf("とても読みづらい");scanf("%d",&data);printf("%d",data);return 0;}
\end{lstlisting}
とても読みづらいですね
他にも例えば
\begin{lstlisting}
if(){

}

if()
{

}
\end{lstlisting}
例えばこのように中括弧の最初の括弧を改行して書くか書かないかなどはプログラマー個人で好みの方を使えます.ただ本当に人によってまちまちなのであまり口を出すと宗教戦争になりかねないので気をつけてください.

\subsection{演習問題}
\begin{enumerate}
	\item 年齢を入力させ、20歳未満なら「酒を飲まないで!」、20歳以上なら「限度をわきまえて!」と出力せよ。
\end{enumerate}
\subsubsection{榊くんの挑戦状}
*榊くんの挑戦状は、BASICから難易度が高いです。もし解けなければ、普通の演習問題に取り組みましょう。\\
\\
問1\\
EASY\\
出題意図:if文,else if文を正しく書けるか。\\

int型の整数a,bを宣言し、入力からa,bをそれぞれ受け取る。\\
$a<b$ならば '<\textbackslash n'を\\
a=bならば '=\textbackslash n'を\\
a>bならば '>\textbackslash n'を\\
出力しなさい。\\
\\
\\
問2\\
BASIC\\
出題意図:if文,else if文の複数条件が書けるか。論理演算子の優先順位を正しく覚えること。\\
\\
int型の整数a,bを宣言し、入力からa,bをそれぞれ受け取る。\\
a<b かつ (100<b または a<100) 	ならば "チャイナ\textbackslash n",\\
a=b かつ (a=100 または b=100) 	ならば "もっちゃん\textbackslash n",\\
a>b かつ (a<100 または b<100) 	ならば "ロブジェ\textbackslash n",\\
上記以外ならば "SAINO\textbackslash n"\\
と出力するプログラムを書きなさい。\\
\\
問3\\
BASIC\\
出題意図:問題文の求めている意味を理解して、if文を上手に使用できるか。\\
\\
4つのint型の変数a,b,c,dを宣言し、入力からa,b,c,dをそれぞれ受け取り、最大値を求めて出力するプログラムを書きなさい。\\
出来た人はプログラムを改造し、最大値と最小値の両方を求めて出力するプログラムにしなさい。\\
\\
なお、この問題は「繰り返し文」「配列」を習えば、N個の要素において最大値,最小値を求めることが容易に書けるようになるので、既に書ける人はfor文,if文,配列等を使用してスマートに解いても良い。\\
\\
問4\\
BASIC\\
出題意図:四則演算を用いて,if文による条件分岐を適用できるか。\\
\\
int型の整数a,b,cが与えられます。\\
a+b=cならば "SAINO\textbackslash n",\\
a-b=cならば "もっちゃん\textbackslash n",\\
a*b=cならば "ロブジェ\textbackslash n",\\
bが0でなく,さらにa/b=cならば "チャイナ\textbackslash n",\\
と出力せよ。なお、a/b において、aもbも整数型ならばその小数点以下は切り捨てられるという仕様もこの機会に認識すること。\\
例えば、a=4,b=3,c=1の時は"もっちゃん"と"チャイナ"が出力される。\\
\\
\\
問題が簡単すぎてつまらない、もうC言語の基本文法は理解してるわボケェ、という人へ\\
\\
*この問題は、現在までの講座内容では解くことが不可能です。進んで勉強をしている人のために作りました。\\
*具体的には、まだ習っていない繰り返し文,条件分岐,配列の知識が必要不可欠です。\\
\\
榊くんの挑戦状EX(未履修範囲の知識が必要な問題)\\
1万個の要素に対して最大値、最小値、平均値、最頻値を求めるプログラムを作成せよ。\\
出来た人には榊くんが晩御飯を奢ります。\\
要素の挿入にはrand()関数を用いて要素を挿入する(N=10000で10000個手入力するのは頭が壊れるのだ)。\\
配列を宣言し、rand()を用いて要素を挿入するプログラムのひな型を下に記す。これが読めなければこの問題は即座にあきらめること。\\
なお、この問題の模範解法は配布しない。\\
講座終了後、知りたければその場でコーディングするので榊まで聞きに来てください。C言語が書けるPCを持参すること。\\
(晩御飯奢り権は新学部1年生先着1名限定)\\
\\
\\
\begin{lstlisting}{EX問題のひな型}
#include<stdio.h>
#include<stdlib.h>
#include<time.h>

#define N 10000

int main(void){
	int a[N];
	srand((unsigned)time(NULL));
	
	for(int i=0;i<N;i++){
		a[i]=rand();
	}
	
	return 0;
}
\end{lstlisting}

\end{document}

