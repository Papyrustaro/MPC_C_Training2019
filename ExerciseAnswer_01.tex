%------------------------------------- ページサイズなどの書式設定
%¥documentclass[a4j,twocolumn, dvipdfmx]{jsarticle} % 二段組の構成にする
%¥documentclass[a4j,notitlepage]{jsarticle} % タイトルだけのページを作らない
\documentclass[a4j,titlepage,dvipdfmx]{jsarticle}   % タイトルだけのページを作る
%------------------------------------- パッケージ読み込み
\newcommand{\stypath}{./sty}
\newcommand{\codepath}{./code/01_Answer}

\usepackage[ipaex]{pxchfon}
%\usepackage{itembkbx}
\usepackage{\stypath/listings}
\usepackage{ascmac}
\usepackage{\stypath/jlisting}
\lstset{% 
showstringspaces=false,%空白文字削除
language={C},% %言語選択
basicstyle={\upshape},% %標準の書体
identifierstyle={\small},% %キーワードでない文字の書体
ndkeywordstyle={\small},% %キーワードその2の書体
stringstyle={\small\ttfamily},% %””で囲まれた文字などの書体
frame={tb},% %枠、デザインなど
breaklines=true,% %行が長くなった時の自動改行
columns=[l]{fullflexible},% %書体による列幅の違いを調整するか
numbers=left,% %行番号を表示するか
xrightmargin=0zw,% %余白の調整?
xleftmargin=0zw,% %余白の調整
numberstyle={\scriptsize},%行番号の書体
stepnumber=1,% %行番号をいくつ飛ばしで表示するか
numbersep=1zw,% %行番号と本文の間隔
morecomment=[l]{//}% 
} 
\title{C言語講座第1回演習問題回答}
\author{那由多}
\date{2019年5月9日}
\begin{document}
\maketitle
\section{問題1}
\subsection{問題1の回答}
\lstinputlisting{\codepath/exe01.c}
\subsection{解説}
We love SAINOという文字列を出力するためにprintf()関数を使っている.\\
また改行するために\textbackslash nというエスケープシークエンスを使っている.\\

\section{問題2}
\subsection{問題2の回答}
\lstinputlisting{\codepath/exe02.c}
\subsection{解説}
数値を入力するのでdoubleで小数点型でnumber変数を宣言する.\\
数字を入力してくださいという文字列を出力した後number変数に数字を入力して,number変数の中身を出力する.\\

\section{問題3}
\subsection{問題3の回答}
\lstinputlisting{\codepath/exe03.c}
\subsection{解説}
半角文字を入力するのでchar型で変数cを宣言する.\\
半角1文字を入力してくださいという文字列を出力した後変数cに半角文字を入力して,変数cの中身を出力する.\\

\section{問題4}
\subsection{問題4の回答}
\lstinputlisting{\codepath/exe03.c}
\subsection{解説}
上底,下底,高さの変数をdoubleで宣言する.\\
上底,下底,高さの順で入力した後計算結果を出力した.\\

このようにprintf()関数の中に計算式を書くことができる.\\
また計算の優先を示す()は()で扱う

()は入れ子構造にすることもできる.\\
\end{document}
