\section{演習問題(さんたろー作)}
\subsection{問1}
\subsubsection{解答例}
\lstinputlisting{\codepath/papyrus/exercise1.c}
\subsubsection{解説}
\begin{verbatim}

https://papyrustaro.hatenablog.jp/entry/2018/12/30/145415
\end{verbatim}

\subsection{問2}
\subsubsection{解答例}
\lstinputlisting{\codepath/papyrus/exercise2.c}
\subsubsection{解説}
\begin{verbatim}

#defineはプログラムの記述自体を書き換えている
意味ではなく、文字列として
MULTの部分を置き換えてから、計算してみよう。
\end{verbatim}

\subsection{問3}
\subsubsection{解答例}
\lstinputlisting{\codepath/papyrus/exercise3.c}
\subsubsection{解説}
\begin{verbatim}
記事の通り。
2年後期の講義でやった内容です。
\end{verbatim}
