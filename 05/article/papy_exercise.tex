\section{演習問題(さんたろー作)}
\subsection{問1(難易度☆☆)}
\subsubsection{問題文}
\begin{verbatim}
自分と敵が交互に攻撃する。自分と敵それぞれの体力と攻撃力を入力し、
自分が先に倒れる場合は「Lose」、敵が先に倒れる場合は「Win」と表示しなさい。
以下にルールと実行例を記す。

1.先攻は自分
2.与えるダメージ = 攻撃者の攻撃力
3.体力が0以下になると倒れる
4.入力は1以上100以下の整数
5.自分の体力、攻撃力、敵の体力、攻撃力の順で入力
\end{verbatim}

\subsubsection{実行例}
\begin{verbatim}
player_hp = 50
player_atk = 30
enemy_hp = 20
enemy_atk = 100
Win!!


player_hp = 50
player_atk = 10
enemy_hp = 30
enemy_atk = 25
Lose!!
\end{verbatim}

\subsection{問2(難易度☆)}
\subsubsection{問題文}
\begin{verbatim}
サンズ君は数学の単位をとれたか確かめたい。
中間テストの点数がa点、期末テストの点数がb点だ。
合格なら「pass」、不合格なら「fail」を入力しなさい。
以下成績の付け方である

1.テストはどちらも100点満点
2.中間テストを4割、期末テストを6割として換算し、その合計を評定とする(100点満点)
3.やさしいので、換算した合計の小数点第一位は切り上げ(59.1→60)
4.60点以上の評定で合格

\end{verbatim}

\subsubsection{実行例}
\begin{verbatim}
a = 65
b = 55
fail

a = 66
b = 55
pass
\end{verbatim}

\subsection{問3(難易度☆)}
\subsubsection{問題文}
\begin{verbatim}
パピルス君はお金をN円持っている。
お店に100円のリンゴがa個、50円のバナナがb本、20円の飴がc個売っている。
パピルス君が手持ちのお金で買える最大個数を出力しなさい。

入力: N, a, b, c(全て0以上1000以下の整数)
出力: 買うことができる最大個数
\end{verbatim}

\subsubsection{実行例}
\begin{verbatim}
n = 1000
a = 10
b = 10
c = 10
23

n = 100
a = 1
b = 1
c = 0
1
\end{verbatim}
