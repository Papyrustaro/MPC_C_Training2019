<<<<<<< HEAD
\section{演習問題(榊作成)}



\subsection{問1 printf文}
\begin{verbatim}
じゃんけんぽん。
YOU LOSE!!!!(ブウウウウウウ!!!!)

俺の勝ち!
何で負けたか、明日まで考えといてください。
そしたら何かが見えてくるはずです。
ほな、いただきます。"
と出力(表示)するプログラムを作成しなさい。
\end{verbatim}
\subsection{問2 入力、四則演算}
\end{verbatim}
整数a,bを入力として受け取り、
a+b、a-b、a*b、b/a、b%aを出力しなさい。
\end{verbatim}

\subsection{問3 if文}
\begin{verbatim}
整数a,bが与えられます。
a<bならば"SAINO",
a=bならば"CHINA",
a>bならば"Lobjet"
と出力しなさい。
\end{verbatim}

\subsection{問4 for文,if文 (または数学) }
\begin{verbatim}
整数x,y(x<y)を入力し、xからyまでの間にある3の倍数,4の倍数の数を求め,
それぞれ表示するプログラムを書きなさい。
\end{verbatim}
\subsection{問5 for文,if文}
\begin{verbatim}
今、n個の点の座標と、整数A,Bが与えられる。
この点の座標(x_i,y_i)が、0<=x_i<=Aと0<=y_i<=Bの両方を満たす時、"Yeah!"と、
そうでなければ"Oh..."を出力せよ。
よくわからなければサンプルを参照せよ。それでもわからなければ質問せよ。
入力形式:
n A B
x_1 y_1
x_2 y_2
.
.
.
x_n y_n

入力:
5 2 4
3 3
2 1
-1 5
1 3
0 0

出力:
Oh...
Yeah!
Oh...
Yeah!
Yeah!
\end{verbatim}

\subsection{問6 配列 for文 if文 数学}
\begin{verbatim}
実数値からなるN個のデータa_1,a_2...a_Nを入力として受け取り、以下のデータを求め、出力せよ。
サンプルは自作し、確かめよ(Nは3とか4でよい)。
1.データの総和
2.データの平均値(相加平均)
3.データの最小値、最大値
\end{verbatim}

\subsection{問7 関数 ポインタ (数学)}
\begin{verbatim}
以下に示す関数を作成せよ。()で囲われた問題は自信のある人だけ解くこと。
1.2つの整数a,bを受け取り、その和を返り値とする関数sum
2.2つの整数a,bを受け取り、その値を入れ替える関数swap(ヒント:ポインタ渡し,参照渡し)
(3.1つの整数xを受け取り、その各桁の和を返り値とする関数を作成せよ。例えば2597の各桁の和は2+5+9+7=23である。)
(4.2つの整数a,bを受け取り、その最大公約数と最小公倍数を出力する関数を作成せよ。)
\end{verbatim}

\subsection{問8 2次元配列 関数 for文 ポインタ}
\begin{verbatim}
9×9の要素を持つ2次元配列を引数として受け取り、その要素を九九の表にする関数を作成せよ。
main関数内で2次元配列を表示し、正しく動作しているかを確認せよ。
\end{verbatim}

\subsection{問9 文字列 for文}
\begin{verbatim}
大文字アルファベッド、小文字アルファベッドからなる文字列Sが与えられる。
この文字列の偶数番目を出力するプログラムを作成せよ。Sの長さは100文字未満であると仮定してよい。
\end{verbatim}

\subsection{問10 文字列 ASCIIコード for文 if文}
\begin{verbatim}
問9同様に、大文字アルファベッド、小文字アルファベッドからなる文字列Sが与えられる。
この文字列の大文字を小文字に、小文字を大文字にして出力するプログラムを作成せよ。
\end{verbatim}

\subsection{問11 ポインタ 文字列 関数}
\begin{verbatim}
文字列を引数として受け取り、
文字列の中身を"aburakatabura"に上書きする関数Aburakataburaを作成せよ。
\end{verbatim}
\subsection{(問12) 解かなくて良いです}
\begin{verbatim}
Nが与えられる。次に、1からNまでの数字と、アルファベッド1文字が、N個与えられる。
例えば以下のように入力が与えられる。
7
1 A
3 C
2 t
7 r
4 o
6 e
5 d
この時、与えられる整数の順番にアルファベッドを出力せよ。
上記の例では"AtCoder"と出力されればよい。
\end{verbatim}




\subsection{問EX 暇を持て余した神々の遊び}

\begin{verbatim}
この大問は、全て解ききって暇すぎるハイレベルな1年生や、暇をしている上級生に対する問題である。
復習問題は余裕で解けるという人には挑んでみて欲しい。
(解けないのが普通で、解けたら強すぎるので競プロをしてください。お願いします。)
解説や解答は記載しないが、榊に聞いてもらえれば解説と解答の実装をやります。結構難しいので頑張ります。
\end{verbatim}

\subsubsection{1.}
\begin{verbatim}
N(N≦500000)個の街を繋げる街道を作りたい。
具体的には、各街から街道を用いて、いくつかの街を通って、すべての街へ行けるようにしたい。
今、街を1,2,...,Nとして、その街i,j同士をつなぐ街道に対するコストC_(i,j)が以下のように推定されている。
                C_(i,j)=i+j
この時、N-1個の街道を用いて、N個の街が、相互に行き来可能となるようにするために必要な、最小のコストを求めなさい。
\end{verbatim}
\subsubsection{2.}
\begin{verbatim}
榊くんは天才なので、神羅万象を全て知り尽くしました。
ところが、脳みその回転速度が非常に遅いため、未来予測をするには向かないようです。
そこで、天才プログラマーであるあなたには、榊くんが解き明かしたルールに基づいて、
ある要素群aについてN年後のものを構築してほしい。与えられるaを0年の状態とする。

要素数Mの、0,1,...,M-1とラベル付けされた要素が、a_1,a_2,...,a_M-1として配置されている。
ここで、1年後には、a_iはa_i番目に移動することがわかっている。例えば、
1 2 0 3
という要素群aは、1年後には
0 1 2 3
となる。

天才プログラマーであるあなたなら、これをシミュレート出来るはずだ。健闘を祈る。
\end{verbatim}










=======
>>>>>>> 6035c97f1d92ddd05867197f47038232d27c08bb
