\section{演習問題(さんたろー作)}
\subsection{最初に}
\begin{verbatim}
今回出す問題は一風変わったものだけ。
基礎を忘れてしまった人は前回までのスライドを見て、過去の演習問題を解いてほしい。

またAtcoderのABCのB問題あたりも復習にはちょうどよいだろう

https://kenkoooo.com/atcoder/#/table/


\end{verbatim}
\subsection{問1(難易度☆☆☆)}
\subsubsection{問題文}
次のプログラムを実行し、なぜそうなるのか考察しなさい
\begin{verbatim}


#include <stdio.h>
int main(void){
  int i = 0;
  for(printf("A"); printf("B"), i++ < 2; printf("C")){
    if(i == 1) continue;
    printf("D");
  }
  return 0;
}
\end{verbatim}

\subsection{問2(難易度☆☆)}
\subsubsection{問題文}
次のプログラムを実行し、なぜそうなるのか考察しなさい
\begin{verbatim}


#include <stdio.h>
#define MULT(x, y) x * y
int main(void){
  double a, b, c, d;
  a = 3; b = 2; c = 1; d = 4;
  printf("%.1f", 2 / MULT(a + b, c + d));
  return 0;
}
\end{verbatim}

\subsection{問3(難易度☆☆☆☆☆)}
\subsubsection{問題文}
以下の記事を参考にして、
Newton-Paphson法を用いた平方根の導出を利用し、
漸化式を用いて円周率を求めるプログラムを書きなさい。\\

http://rabbitfoot141.hatenablog.com/entry/2017/04/28/123137
