%------------------------------------- ページサイズなどの書式設定
%¥documentclass[a4j,twocolumn, dvipdfmx]{jsarticle} % 二段組の構成にする
%¥documentclass[a4j,notitlepage]{jsarticle} % タイトルだけのページを作らない
\documentclass[a4j,titlepage,dvipdfmx]{jsarticle}   % タイトルだけのページを作る
%------------------------------------- パッケージ読み込み
\newcommand{\stypath}{./sty}
\newcommand{\codepath}{./code}
\newcommand{\articlepath}{./article}
\usepackage[ipaex]{pxchfon}
%\usepackage{itembkbx}
\usepackage{\stypath/listings}
\usepackage{ascmac}
\usepackage{\stypath/jlisting}
\lstset{% 
showstringspaces=false,%空白文字削除
language={C},% %言語選択
basicstyle={\upshape},% %標準の書体
identifierstyle={\small},% %キーワードでない文字の書体
ndkeywordstyle={\small},% %キーワードその2の書体
stringstyle={\small\ttfamily},% %””で囲まれた文字などの書体
frame={tb},% %枠、デザインなど
breaklines=true,% %行が長くなった時の自動改行
columns=[l]{fullflexible},% %書体による列幅の違いを調整するか
numbers=left,% %行番号を表示するか
xrightmargin=0zw,% %余白の調整?
xleftmargin=0zw,% %余白の調整
numberstyle={\scriptsize},%行番号の書体
stepnumber=1,% %行番号をいくつ飛ばしで表示するか
numbersep=1zw,% %行番号と本文の間隔
morecomment=[l]{//}% 
} 
\title{C言語講座第九回・構造体}
\author{MPC部員}
\date{2019年7月4日}
\begin{document}
\maketitle
\section{構造体}
構造体を用いることによって、複数の異なる型をまとめて用いることが出来ます。
この構造体はゲームのキャラクターのデータをつくるときなどに使われます。
まず以下の例を見てみてましょう。


\lstinputlisting{\codepath/10-1.c}
\begin{itembox}{出力結果}
HP:50 AP:30 MP:30
\end{itembox}
構造体の型を宣言するためにはstructというものをつけます。そのあとに変数のルールと同様に構造体の
型名を付けます。型名が構造体タグと呼ばれるものです。構造体タグも main 関数の前に書くというのが普通
です。
上の例では struct\_player と構造体タグを指定していましたがtypedefというものを使うことによって新し
い型を自分で宣言するということが出来るようになります。

\lstinputlisting{\codepath/10-2.c}

構造体のポインタを使うときには*ではなく- \textgreater を使います。
\lstinputlisting{\codepath/10-3.c}

構造体メンバにchar型のnameのような名前のメンバを追加することでplayer名も敢為することが出来るようになったりします。

\section{列挙体}
列挙体を使うと、値に特別な意味を持たせることが出来ます。
列挙体を使うにはenumを使います。


\lstinputlisting{\codepath/10-4.c}
\begin{itembox}{出力結果}
0 1 2 3 4 5 6
\end{itembox}
また以下のような使いかたもします。

\lstinputlisting{\codepath/10-5.c}
中身は自分でいじってみたりして色々試してみましょう。


%構造体と列挙体
\section{プリプロセッサ}
%プリプロセッサ
\end{document}
