\section{榊作成 演習問題}
\begin{verbatim}
「動作を確認せよ。」というのは、関数の返り値をmain関数で出力せよという意味です。
EX問題は大不評のため、廃止されました。
これからはおまけ問題へと名前を変更し、講座内容にあった問題になります。
よって、解答は配布されます。
問題文中のa,b,x,y,PRINTなどは、テキトーに決めた名前です。
やりづらければ自分で変数名,関数名を決めて良いです。
\end{verbatim}

\subsection{榊くんのぼやき}
一般に、再帰関数は、再帰関数を使わないで実装することが可能です。
なので、今回の講座で出てきたような再帰関数は、実際には使わないことの方が多いです。
しかし、再帰関数を使うことで膨大な量のソースコードを書かないで済むことがあるので、再帰関数を理解していると、効率的にソースコードが解析できる場合があります。

おまけ問題に、再帰関数を使うことでソースコードのサイズが数分の1になると推測される問題を用意しました。一般に、深さ優先探索と呼ばれる手法で解ける問題です。配列を引数にするには、ポインタの概念を知っている必要があるので、今回は配列をグローバル変数にして実装しましょう。
難しいのでスルーすることを推奨します。学部2年でも演習時間内に書ける人は1割程度だと思います。

\subsection{問1}
\begin{verbatim}
「That was the worst night's sleep I've ever had.」
と出力するvoid型の関数PRINTを作成し、main関数から呼び出して、動作を確認せよ。
どうでもよいが、日本語訳は「(昨夜の眠りは)人生で最悪の眠りだった。」である。
\end{verbatim}

\subsection{問2}
\begin{verbatim}
呼び出すと114514を返すint型の関数HOMOを作成せよ。┌(^o^┐)┐...
\end{verbatim}

\subsection{問3}
\begin{verbatim}
double型の引数a,bを受け取り、
その和a+bを返り値とするdouble型の関数SUMを作成し、
main関数から呼び出して、動作を確認せよ。
\end{verbatim}

\subsection{問4}
\begin{verbatim}
int型の引数x,yを受け取り、
その積を返り値とするint型の関数MULTIPLICATIONを作成し、
main関数から呼び出して、動作を確認せよ。
\end{verbatim}

\subsection{問5}
\begin{verbatim}
int型の引数a,bを受け取り、
大きい方の値を返り値とするint型の関数MAXを作成し、
main関数から呼び出して、動作を確認せよ。
\end{verbatim}

\subsection{問6}
\begin{verbatim}
以下のコードはコンパイルが通らない。その理由をスコープの概念から理解せよ。
また、コンパイルが出来るように書き換えよ。
ただし、MAX関数の返り値がa,bの最大値であることと、
それをmain関数で出力することを変更してはいけない。
例えばグローバル変数を利用してもよいし、MAX関数が引数を受け取るようにしても良い。
\end{verbatim}

\begin{verbatim}
#include<stdio.h>

int MAX();

int main(void){
	int a,b;
	scanf("%d%d",&a,&b);

	printf("%d",MAX());

	return 0;
}

int MAX(){
	if(a<b){return b;}
	else{return a;}
		
}
\end{verbatim}

\subsection{問7}
\begin{verbatim}
int型の引数nを受け取り、n回
「These maggots are rich in protein.」
と出力するvoid型の関数PRINTを作成し、main関数から呼び出して動作を確認せよ。
どうでもよいが、日本語訳は「ウジ虫は栄養が豊富だ。」です。
\end{verbatim}

\subsection{問8}
\begin{verbatim}
int型の引数nを受け取り、
その階乗(n!)を返り値とするint型の関数FUCTORIALを作成し、
main関数から呼び出して出力せよ。
再帰関数でも良いし、for文を回して求めてもよい。
nが大きいとオーバーフローするが、それも気にしないで良い。
\end{verbatim}

\subsection{問9}
\begin{verbatim}
ユークリッドの互除法は、
下記のようなC言語による実装で求めることが出来る。
//////で囲われた部分が、
実際にユークリッドの互除法をして解を表示している部分である。
ここで、int型のa,bは、2つの自然数であり、
printfで表示されているbは、
ユークリッドの互除法によって求められた2つの自然数の最大公約数である。
下記のコードを参考に(コピペしても良い)、
ユークリッドの互除法によって最大公約数を求める関数GCDを作成せよ。
ただし、返り値を最大公約数として、main関数から呼び出して動作を確認すること。
興味があれば、再帰関数で作成する方法も調べて実装してみよ。
\end{verbatim}

\begin{verbatim}
#include<stdio.h>

int main(void){
	int a,b,r;
	scanf("%d%d",&a,&b);
/////////////////////////////
	r=a%b;

	while(r!=0){
		a=b;
		b=r;
		r=a%b;
	}

	printf("%d",b);
/////////////////////////////
	return 0;
}
\end{verbatim}

\section{追伸}
\begin{verbatim}
来週の講座でポインタの概念を習うと、↓の2つのような問題が解けます。
今までのC講座の知識では解けないです。

来週への伏線問題(解答は配布されません)


double型の引数a,bを受け取り、
その値を入れ替えるvoid型の関数を作成し、
main関数から呼び出して、動作を確認せよ。


int型の配列aと、その要素数nを受け取り、
配列の総和を返り値とするint型の関数を作成し、
main関数から呼び出して動作を確認せよ。

\end{verbatim}

\section{おまけ問題(問10)}
\begin{verbatim}
9本の木材があります。今、各木材はl_1,l_2,…,l_9という長さです。
これらの木材から長さがA,B,Cとなる木材が欲しいのですが、
チェーンソーも接着剤もありません。

さて、榊くんは超能力者なので、
以下のような3つの力を任意の順番で何回でも使えます。
・体力を1消耗し、任意の木材1本の長さを1だけ増やす。
・体力を1消耗し、長さが2以上の任意の木材1本の長さを1だけ減らす。
・体力を10消耗し、任意の木材2本を選んで継ぎ目なく接続し、
1本の木材とする。この時、新しく出来た1本の木材は、
選ばれた2本の木材の長さの和となる。(新しくできた木材にも、力が使える。)

上記の力を持っている榊くんですが、余計な体力は使いたくありません。
長さA,B,Cとなる木材を作成するのに必要な最小の体力を求めてください。
なお、体力は無尽蔵にあるものとし、
1≦l_1,l_2,…l_9,A,B,C≦1000かつ、A>B>Cを満たす入力が与えられるものとする。

(任意の木材の長さを1増やす、1減らす力があるので、必ず解は存在します。)
(再帰を使わないで書くのはメンドクサイので解答例は再帰関数で実装しています。)
(あり得る組み合わせを全探索して、最小値を出力します。)

以下、入力例と出力例

200 130 90
100 40 50 200 150 130 20 70 90

answer=0

30 30 30
100 40 50 200 150 130 20 70 90

answer=40

900 575 400
135 318 614 271 481 12 38 29 288 57

answer=73

962 714 195
158 217 739 217 19 482 182 927 662

answer=53
\end{verbatim}
