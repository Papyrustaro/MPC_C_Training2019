\section{演習問題(さんたろー作)}
\subsection{問1(難易度☆)}
\subsubsection{問題文}
\begin{verbatim}
2つの整数のうち大きいほうの値を返す関数を作成し、挙動を確かめなさい。
ただし、同値の場合は最初の値(a)を返すものとする。

返り値: int(大きい方の値)
引数: int a(1つ目の整数), int b(2つ目の整数)
(-1000 <= a, b <= 1000)
\end{verbatim}

\subsubsection{実行例}
\begin{verbatim}
a = 5
b = 10
最大値は10です

a = -10
b = -100
最大値は-10です
\end{verbatim}

\subsection{問2(難易度☆)}
\subsubsection{問題文}
\begin{verbatim}
AのN乗の値を返す関数を作成し、挙動を確かめなさい。
(基本はループ文でやるが、再帰関数を利用してもよい)
多少の誤差は許容する。

返り値: double(aのn乗の値)
引数: double a, int n
(0 < a < 100, 0 <= n <= 10)
\end{verbatim}

\subsubsection{実行例}
\begin{verbatim}
a = 10
n = 3
10.000000の3乗は1000.000000です

a = 2.5
n = 3
2.500000の3乗は15.625000です
\end{verbatim}

\subsection{問3(難易度☆)}
\subsubsection{問題文}
\begin{verbatim}
char型の変数1つを引数としてもち、
アルファベットの大文字なら小文字に、小文字なら大文字に、
アルファベット以外の文字ならそのまま返す関数を作成し、挙動を確かめなさい。

返り値: char(変換した文字)
引数: char(変換する文字)
\end{verbatim}

\subsubsection{実行例}
\begin{verbatim}
c = c
変換後: C

c = A
変換後: a

c = !
変換後: !
\end{verbatim}
