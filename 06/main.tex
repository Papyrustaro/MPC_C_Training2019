%------------------------------------- ページサイズなどの書式設定
%¥documentclass[a4j,twocolumn, dvipdfmx]{jsarticle} % 二段組の構成にする
%¥documentclass[a4j,notitlepage]{jsarticle} % タイトルだけのページを作らない
\documentclass[a4j,titlepage,dvipdfmx]{jsarticle}   % タイトルだけのページを作る
%------------------------------------- パッケージ読み込み
\newcommand{\stypath}{./sty}
\newcommand{\codepath}{./code}
\newcommand{\articlepath}{./article}
\usepackage[ipaex]{pxchfon}
%\usepackage{itembkbx}
\usepackage{\stypath/listings}
\usepackage{ascmac}
\usepackage{\stypath/jlisting}
\lstset{% 
showstringspaces=false,%空白文字削除
language={C},% %言語選択
basicstyle={\upshape},% %標準の書体
identifierstyle={\small},% %キーワードでない文字の書体
ndkeywordstyle={\small},% %キーワードその2の書体
stringstyle={\small\ttfamily},% %””で囲まれた文字などの書体
frame={tb},% %枠、デザインなど
breaklines=true,% %行が長くなった時の自動改行
columns=[l]{fullflexible},% %書体による列幅の違いを調整するか
numbers=left,% %行番号を表示するか
xrightmargin=0zw,% %余白の調整?
xleftmargin=0zw,% %余白の調整
numberstyle={\scriptsize},%行番号の書体
stepnumber=1,% %行番号をいくつ飛ばしで表示するか
numbersep=1zw,% %行番号と本文の間隔
morecomment=[l]{//}% 
} 
\title{C言語講座第六回}
\author{MPC部員}
\date{2019年6月27日}
\begin{document}
\maketitle
\section{関数について}
%関数についての説明
\section{再帰関数について}
\subsection{再帰関数}

 関数は処理中に自分自身を呼び出すことができます。
そしてそれを再帰呼び出しといいます。以下のプログラムでは例として階乗の計算を行っています。
\lstinputlisting{\codepath/func9.c}
factorial関数内の処理を見ると自分自身を呼び出しています。再帰関数はループ文と同様に終了処理を与えないと無限ループになったりするので注意してください。
今回はfactorial(10)より
10!=10*(9!)   9! =9*(8!)    8!=8(7!)…のようになっています。

今回はxが0となると1を返すようになっています。0!=1ということです。
ちなみに勘の鋭い方は"これループでいいじゃん"って思いますよね。その通りです。余裕があれば書いてみてください。
%再帰関数についての説明(難しいので分けました)
\section{関数のスコープについて}
%変数のスコープについての説明
\section{復習問題}
%演習問題
\end{document}
