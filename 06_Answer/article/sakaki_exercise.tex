\section{演習問題}
\subsection{問1}
\subsubsection{問題}
\lstinputlisting{\codepath/sakaki/answer1.c}
\subsubsection{解説}

英文をそのまま出力するだけのprintf()を用意して、それだけが入っている関数を作ればよい。
引数は必要がないので何も書かなくて良い。

\section{演習問題}
\subsection{問2}
\subsubsection{問題}
\lstinputlisting{\codepath/sakaki/answer2.c}
\subsubsection{解説}

返り値が114514、つまりreturn 114514;のみが書かれた関数を作成すればよい。
引数は必要がないので何も書かなくて良い。

\section{演習問題}
\subsection{問3}
\subsubsection{問題}
\lstinputlisting{\codepath/sakaki/answer3.c}
\subsubsection{解説}

double型の引数a,bを受け取り、その和を返り値とする。つまりreturn a+b;が書かれた関数を作成すればよい。

\section{演習問題}
\subsection{問4}
\subsubsection{問題}
\lstinputlisting{\codepath/sakaki/answer4.c}
\subsubsection{解説}

int型の引数x,yを受け取り、その積を返り値とする。つまりreturn x*y;が書かれた関数を作成すればよい。

\section{演習問題}
\subsection{問5}
\subsubsection{問題}
\lstinputlisting{\codepath/sakaki/answer5.c}
\subsubsection{解説}

int型の引数a,bを受け取り、大きい方の値を返す。if文を用いて、
a<bならばb,それ以外ならばaを返り値とすればよい。

\section{演習問題}
\subsection{問6}
\subsubsection{問題}
\lstinputlisting{\codepath/sakaki/answer6-1.c}
\lstinputlisting{\codepath/sakaki/answer6-2.c}
\subsubsection{解説}

まず、main関数内で宣言されたint型のa,bは、ローカル変数なので、main関数内でしか使えない。
6-1は関数の引数を使わず、int a,bをグローバル変数とした実装である。
クソコードと言われるので絶対にやめましょう。
6-2はmain関数内でint a,bを宣言し、関数に引数として渡している。こちらのコードを推奨する。

\section{演習問題}
\subsection{問7}
\subsubsection{問題}
\lstinputlisting{\codepath/sakaki/answer7.c}
\subsubsection{解説}

int型の引数nを受け取り、for文でn回だけprintf()すればよい。

\section{演習問題}
\subsection{問8}
\subsubsection{問題}
\lstinputlisting{\codepath/sakaki/answer8-1.c}
\lstinputlisting{\codepath/sakaki/answer8-2.c}
\subsubsection{解説}

8-1はfor文によってnから1までを掛け算している。
8-2は、再帰関数を用いてn*(n-1)*...*1を計算して返り値としている。
どちらでやっても良い。基本的に再帰関数は使わなくても実装は可能。

\section{演習問題}
\subsection{問9}
\subsubsection{問題}
\lstinputlisting{\codepath/sakaki/answer9-1.c}
\lstinputlisting{\codepath/sakaki/answer9-2.c}
\subsubsection{解説}

9-1はwhile文によって実装している。
9-2は再帰関数によって実装している。
アルゴリズム自体は高校数学を復習して頂きたい。

\section{演習問題}
\subsection{問10}
\subsubsection{問題}
\lstinputlisting{\codepath/sakaki/answer10.c}
\subsubsection{解説}

各木材に対して、
\begin{enumerate}
\item 何もしない
\item 大きさを1増やす
\item 大きさを1減らす
\item 2本を1本にまとめる
\end{enumerate}
という操作が行える。これを木材9本に対して行えばよい。さらによく考えてみると、2本にまとめる全ての組み合わせに対して、最後に+1,-1を必要な回数だけ行えばよいことに気付く。
よって、
\begin{enumerate}
\item 何もしない
\item Aに$l_i$を合成
\item Bに$l_i$を合成
\item Cに$l_i$を合成

\end{enumerate}
の4通りに対して、最後に+1,-1を必要な分行うことで、この問題を解くことが出来る。
再帰関数を使うと簡単に書くことが出来る。MIN,ABS関数はmath.hをインクルードすることでmin,abs関数として使用できる。
