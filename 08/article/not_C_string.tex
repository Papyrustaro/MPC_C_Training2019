\section{おまけ:C言語以外で文字列を扱う場合}
C言語はみなさんここまででよくわかったと思いますが文字列操作が得意ではありません.\\
そこで文字列処理がC言語より比較的容易なC++で文字列処理を書いてみました.
\lstinputlisting{\codepath/cpp_string.cpp}
\begin{itembox}[]{実行結果}
\begin{verbatim}
5
Yeah!aaa!!!
\end{verbatim}
\end{itembox}
先ほどより多少はましでしょうかね.

今度は文字列処理が比較的得意なpythonで一度先ほどのプログラムを書いてみましょう.
\lstinputlisting{\codepath/python_string.py}
\begin{itembox}[]{実行結果}
\begin{verbatim}
5
Yeah!aaa!!!
\end{verbatim}
\end{itembox}

文体は違いますがこれでも同じように動きます.
C言語に比べてpythonは学習が比較的容易だと言われています.
システム理化学科の人はおそらく後期に行いますが,MPCでもpython講座を開くか考えているらしいので学んでみるのも良いかと思います.
