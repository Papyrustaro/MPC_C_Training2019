\section{前回の復習}
\subsection{ポインタ}
アドレスは定数が保存されている場所の番号のようなものです。アドレスを参照する場合は変数か型に*をつけることででポインタ変数にする必要があります。
ポインタとはその場所を保存するものです。また、ポインタを有効に活用することで、
関数間での変数の値の変更ができたり、配列を扱うことができます。

\begin{itembox}{ポインタの宣言方法}
\begin{verbatim}
型名 *ポインタ名
///
型名* ポインタ名

例)int *num ;
   int* num,num2;
\end{verbatim}
\end{itembox}
このときの注意なのですが例で示した下の方は2個目の変数はポインタ変数ではありません。実はnum2は通常の変数を取ります。わかりずらいので基本上の型宣言がいいと思います。
\begin{table}[htb]
\begin{center}
\begin{tabular}{|c|c|}
\hline
0018FF4C & a[0]\\ \hline
0018FF4D & a[1]\\ \hline
0018FF4E & a[2]\\ \hline
0018FF4F & a[3]\\ \hline\hline\hline
0018FF60 & \&a[0]\\
\hline

\end{tabular}
\caption{ int型の変数の格納例}
\end{center}
\end{table}
この場合0018FF60には\&a[0]、つまりa[0]のアドレス0018FF4Cが入っています。
\subsection{復習問題}
\noindent
1.変数aを宣言し、そのアドレスを表示せよ\\
2.2つの変数の値を入れ替えるvoid型のswap関数を作成し,変数a,bを入れ替えよ.\\
分からない場合はどんどん前回資料や末尾の参考文献などを参照してみてください。


