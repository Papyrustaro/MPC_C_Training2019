\section{榊作成 演習問題}
%\begin{verbatim}
榊くんのぼやき


 C言語の文字列やりたくねえええええええええええええええええええええええええええええええええええええええええええええええええええええええええええええええええええええええええええええええええええええええぇぇぇぇぇぇぇぇぇえぇぇぇぇぇぇぇぇぇぇえぇえぇぇぇぇぇぇぇぇ!!!!!!!!!!!!!!!!!!!!!!!!!!!!!!!!!!!!!!!!!!!!!!!!
 
 
それはおいておいて、文字列の操作が必要ならばstring.hにある関数の使い方を勉強したほうがいいよ。
俺はincludeするのとその使い方を調べるのがメンドクサイので、解説では自分で関数を書いたけど...。
%\end{verbatim}

\subsection{問1}
\begin{verbatim}
以下のコードは、scanf,fgetsを用いて、200文字未満の任意の文字列を受け取り、
そのまま画面に表示させるプログラムである。

#include<stdio.h>

int main(void){
    char str_fgets[200];
	char str_scanf[200];
	
    fgets(str_fgets,sizeof(str_fgets),stdin);
    scanf("%s",str_scanf);
	
    printf("%s",str_fgets);
    printf("%s",str_scanf);
	
    return 0;
}

以下の問題文の通りに試して、その違いを確認してほしい。
このプログラムに以下のような入力をせよ。
WE1love2SAINO3
WE1love2SAINO3
fget,scanf共に正しく出力されるはずだ。
fgetsでは改行文字も文字列に格納されていること、
scanfでは改行文字を格納していないことを確認してほしい。

以下の入力は、最初に空白文字が2つ、文字列中に何個か空白文字がある例である。
scanfでは、まず空白文字か改行文字でないところから読み取りをはじめ、
空白文字か改行文字になるまで読み取るので、正しく入力されていないことを確認してほしい。
  I'm a green coder.
  I'm a green coder.
出力結果は以下のようになるはずだ。
  I'm a green coder.
I'm
イマイチわからなければ先輩に聞こう。解説は問題文の通りなので解答pdfには記載しない。
これで問1は終わりである。
\end{verbatim}

\subsection{問2}
\begin{verbatim}
C言語において、半角英数文字はASCIIコードに従い、整数値として格納されている。
以下のコードは、文字を格納しているchar型の変数を、整数値として出力しているものである。
#include<stdio.h>

int main(void){
        char cA='A';
        char cZ='Z';
        char ca='a';
        char cz='z';
	
        printf("%d %d %d %d\n",cA,cZ,ca,cz);
	
        return 0;
}
65 90 97 122と出力されるはずだ。
56から90までにAからZ、97から122までにaからzがASCIIコードに充てられている。

この値をよく見ると、大文字を小文字にしたければ+32すれば良く、
小文字を大文字にしたければ-32すれば良いことに気付けるだろう。
(実際使用する時には'a'-'A'という書き方で32を表現することが多いが。)

いずれ他の言語で文字列を扱う際には全く役に立たないが、C言語では覚えておくと多少役に立つ。
検索用ワード:アスキーコード
これで問2は終わりである。解説pdfには解説を記載しない。
\end{verbatim}

\subsection{問3}
\begin{verbatim}
空白も存在する文字列を入力し、その奇数文字目を出力するプログラムを作成せよ。
文字列の長さは50文字未満であると仮定してよい。
うまく実装しないと末尾に謎の文字列も出力されるが、今はそれでもかまわない。
出来なければ解説を見て理解してほしい。
例えば、
I like Competitive programming.
を入力すると、
Ilk opttv rgamn.
が出力されるプログラムを作成せよ。
\end{verbatim}

\subsection{問4}
\begin{verbatim}
大文字と小文字のアルファベットからなる文字列が与えられる。
これには空白文字は含まれず、長さは40未満であると仮定してよい。
問2を参考に、大文字は小文字に、小文字は大文字に変換して出力しなさい。
上手く実装しないと末尾に謎の文字列が出力されるかもしれないが、今はそれでも良い。
例えば
AbcdEfgHikjLMn
を入力すると
aBCDeFGhIKJlmN
が出力されるプログラムを作成せよ。
\end{verbatim}

\subsection{おまけ}
\href{https://algorithm-visualizer.org/divide-and-conquer/quicksort}{algorithm-visualizer}\newline
このサイト滅茶苦茶面白い(個人差アリ)ので興味があったら見てね。上の方のplayで動くよ。