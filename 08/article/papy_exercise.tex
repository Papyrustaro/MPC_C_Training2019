\section{演習問題(さんたろー作)}
\subsection{最初に}
\begin{verbatim}
二重ポインタはわかりづらく、使う機会も少ないはず。
先に文字列の理解を深めるのが妥当だと思います。
関数など前回までの基本がわからない場合は
以前のスライドや演習問題を振り返りましょう。

全体を通しての参考文献(二重ポインタ、文字列)
https://9cguide.appspot.com/15-07.html
https://9cguide.appspot.com/14-01.html
http://www.c-tipsref.com/reference/string.html
\end{verbatim}
\subsection{問1(難易度☆)}
\subsubsection{問題文}
\begin{verbatim}
以下のプログラムを実行し、なぜこのような結果になるのか考えなさい。
また、viewArray関数のコメントを外して実行してみなさい。
\end{verbatim}

\subsubsection{コード}
\begin{verbatim}
#include <stdio.h>

void viewArray(int a[][3], int size);

int main(){
  int a[2][3] = {{1, 2, 3}, {4, 5, 6}};

  for(int i = 0; i < 2; i++){
    for(int j = 0; j < 3; j++){
      printf("a[%d][%d] = %d\n", i, j, *(*(a + i) + j));
    }
  }
  //viewArray(a, sizeof(a) / sizeof(a[0]));
  return 0;
}

void viewArray(int a[][3], int size){
  for(int i = 0; i < size; i++){
    for(int j = 0; j < 3; j++){
      printf("a[%d][%d] = %d\n", i, j, *(*(a + i) + j));
    }
  }
}
\end{verbatim}

\subsection{問2(難易度☆)}
\subsubsection{問題文}
\begin{verbatim}
以下のプログラムを実行し、なぜこのような結果になるのか考えなさい。

ASCIIコード表↓
http://www3.nit.ac.jp/~tamura/ex2/ascii.html
\end{verbatim}

\subsubsection{コード}
\begin{verbatim}
#include <stdio.h>

int main(){
  char aLow = 'a', aUp = 'A';
  printf("%cの文字コードは%dです\n", aLow, aLow);
  printf("%cの文字コードは%dです\n", aUp, aUp);

  printf("%c + 1 = %c\n", aLow, aLow + 1);
  printf("%c - 1 = %c\n", aLow, aLow - 1);

  printf("a - A = %d\n", 'a' - 'A');
  return 0;
}
\end{verbatim}

\subsection{問3(難易度☆☆)}
\subsubsection{問題文}
\begin{verbatim}
(この問題は前にも出しましたが、やってる人少ないと思うのでもう一回)
char型の変数1つを引数としてもち、
アルファベットの大文字なら小文字に、小文字なら大文字に、
アルファベット以外の文字ならそのまま返す関数を作成し、挙動を確かめなさい。

返り値: char(変換した文字)
引数: char(変換する文字)
\end{verbatim}

\subsubsection{実行例}
\begin{verbatim}
c = c
変換後: C

c = A
変換後: a

c = !
変換後: !
\end{verbatim}

\subsection{問4(難易度☆)}
\subsubsection{問題文}
\begin{verbatim}
以下のプログラムを実行し、なぜこのような結果になるのか考えなさい。
\end{verbatim}

\subsubsection{コード}
\begin{verbatim}
#include <stdio.h>
#include <string.h>
#define N 256

int main(){
  char s1[N] = "Under";
  char s2[N] = "tale";

  //strlen
  for(int i = 0; i < (int)strlen(s1); i++){
    printf("s1[%d] = %c\n", i, s1[i]);
  }

  //strcmp
  if(strcmp(s1, s2) > 0){
    printf("%sのほうが%sより文字コード順で大きい\n", s1, s2);
  }else if(strcmp(s1, s2) == 0){
    printf("%sと%sは同じ\n", s1, s2);
  }else{
    printf("%sのほうが%sより文字コード順で小さい\n", s1, s2);
  }

  //strcat
  strcat(s1, s2);
  printf("s1 = %s\n", s1);

  //strstr
  char *t = strstr(s1, "er");
  if(t != NULL){
    printf("s1はerを含む\n");
    printf("t = %s", t);
  }else{
    printf("s1はerを含まない\n");
  }

  return 0;
}
\end{verbatim}

\subsection{問5(難易度☆☆)}
\subsubsection{問題文}
\begin{verbatim}
以下のプログラムを実行し、なぜこのような結果になるのか考えなさい。
数字に限らず、様々な入力をしなさい。

以下の記事参考↓
https://papyrustaro.hatenablog.jp/entry/2019/01/18/163521
https://9cguide.appspot.com/06-02.html
\end{verbatim}

\subsubsection{コード}
\begin{verbatim}
#include <stdio.h>
#include <stdlib.h>
#include <string.h>
#define N 256

void trimEnd(char str[]){
  int i = 0;
  while(str[i] != '\n' && i < (int)strlen(str)){
    i++;
  }
  str[i] = '\0';
}

int main(void){
  char s[N];

  do{
    printf("0以上100以下の番号を入力してください: ");
    fgets(s, N, stdin);
    trimEnd(s);
    printf("番号 = %d\n", atoi(s));
  }while((atoi(s) == 0 && strcmp(s, "0") != 0) || atoi(s) < 0 || atoi(s) > 100);
  printf("\n入力成功\n");

  return 0;
}
\end{verbatim}
