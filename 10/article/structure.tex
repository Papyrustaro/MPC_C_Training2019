\section{構造体}
構造体を用いることによって、複数の異なる型をまとめて用いることが出来ます。
この構造体はゲームのキャラクターのデータをつくるときなどに使われます。
まず以下の例を見てみてましょう。


\lstinputlisting{\codepath/10-1.c}
\begin{itembox}{出力結果}
HP:50 AP:30 MP:30
\end{itembox}
構造体の型を宣言するためにはstructというものをつけます。そのあとに変数のルールと同様に構造体の
型名を付けます。型名が構造体タグと呼ばれるものです。構造体タグも main 関数の前に書くというのが普通
です。
上の例では struct\_player と構造体タグを指定していましたがtypedefというものを使うことによって新し
い型を自分で宣言するということが出来るようになります。

\lstinputlisting{\codepath/10-2.c}

構造体のポインタを使うときには*ではなく- \textgreater を使います。
\lstinputlisting{\codepath/10-3.c}

構造体メンバにchar型のnameのような名前のメンバを追加することでplayer名も敢為することが出来るようになったりします。

\section{列挙体}
列挙体を使うと、値に特別な意味を持たせることが出来ます。
列挙体を使うにはenumを使います。


\lstinputlisting{\codepath/10-4.c}
\begin{itembox}{出力結果}
0 1 2 3 4 5 6
\end{itembox}
また以下のような使いかたもします。

\lstinputlisting{\codepath/10-5.c}
中身は自分でいじってみたりして色々試してみましょう。


