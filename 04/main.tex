%------------------------------------- ページサイズなどの書式設定
%¥documentclass[a4j,twocolumn, dvipdfmx]{jsarticle} % 二段組の構成にする
%¥documentclass[a4j,notitlepage]{jsarticle} % タイトルだけのページを作らない
\documentclass[a4j,titlepage,dvipdfmx]{jsarticle}   % タイトルだけのページを作る
%------------------------------------- パッケージ読み込み
\newcommand{\stypath}{./sty}
\newcommand{\codepath}{./code}
\newcommand{\articlepath}{./article}
\usepackage[ipaex]{pxchfon}
%\usepackage{itembkbx}
\usepackage{\stypath/listings}
\usepackage{ascmac}
\usepackage{\stypath/jlisting}
\lstset{% 
showstringspaces=false,%空白文字削除
language={C},% %言語選択
basicstyle={\upshape},% %標準の書体
identifierstyle={\small},% %キーワードでない文字の書体
ndkeywordstyle={\small},% %キーワードその2の書体
stringstyle={\small\ttfamily},% %””で囲まれた文字などの書体
frame={tb},% %枠、デザインなど
breaklines=true,% %行が長くなった時の自動改行
columns=[l]{fullflexible},% %書体による列幅の違いを調整するか
numbers=left,% %行番号を表示するか
xrightmargin=0zw,% %余白の調整?
xleftmargin=0zw,% %余白の調整
numberstyle={\scriptsize},%行番号の書体
stepnumber=1,% %行番号をいくつ飛ばしで表示するか
numbersep=1zw,% %行番号と本文の間隔
morecomment=[l]{//}% 
} 
\title{C言語講座第四回}
\author{MPC部員}
\date{2019年5月23日}
\begin{document}
\maketitle
\section{配列}
 今までは、変数を使うときは一つずつ宣言してきました。ですが、100個、1000個などと個数の多いデータを取り扱いたいとき
に変数をそのぶん宣言するのは無理があります。
そんなときに便利なのが配列です。配列を使うと複数のデータをまとめて取り扱うことができます。
\subsection{配列の宣言について}
ということで今紹介した配列の宣言方法についての説明です。宣言方法は以下のようになっています。
\begin{itembox}{配列の宣言方法}
\begin{verbatim}
型名 配列名 [要素数];
例)int arry[5];
\end{verbatim}
\end{itembox}
配列名というのは、変数の名前と同じと考えて大丈夫です。要素数というのは、つくられる変数の数のことです。
ここで要素数として宣言できるのは自然数のみです。
\subsection{配列の扱い方例}
 配列の扱い方として以下のプログラムを実行してみましょう。
\lstinputlisting{\codepath/array1.c}
\begin{itembox}{実行結果}
array[0]=5\\
array[1]=6\\
array[2]=7
\end{itembox}
上のプログラムを見てわかるように、配列の要素数は0から始まります。つまり、arra[3]と宣言するとarray[0]~array[2]までがつくられるということです。

\subsection{配列の初期化、代入について}
 配列も宣言と同時に初期化を行うことができます。\\
以下のプログラムを実行してみましょう。実行結果は前のと同じなので省略します。
\lstinputlisting{\codepath/array1-1.c}

ちなみに以下のように宣言と同時に代入を行う場合、配列の要素数を省略することができます。
また、for文を利用して、配列の中身を以下のように確認することができます。これも実行結果は同じになります。

\lstinputlisting{\codepath/array2.c}
以下のような代入はエラーとなります。注意するようにしてください。
\lstinputlisting{\codepath/error.c}

ちなみに以下のようなこともできますわ。
\lstinputlisting{\codepath/array3.c}
\begin{itembox}{実行結果}
5 numbers put on please\\
5 6 7 8 9\\
5 6 7 8 9
\end{itembox}


\section{2次元配列}
 1次元配列はary[1]のように添え字が1つでした。2次元配列は単純にary[1][1]のように添え字が2つになった配列のことを言います。
ちなみにこれはオセロや五目並べの盤面、グリッドなどをつくるときによく使います。

\subsection{2次元配列の宣言}
 まずは2次元配列の宣言の仕方です。宣言方法は以下のようになっています。

\begin{itembox}{2次元配列の宣言方法}
型名 配列名[要素数][要素数];\\
例)int array[3][4];
\end{itembox}

ちなみに2次元配列にも次のように代入できます。
\lstinputlisting{\codepath/doublearray1.c}

\begin{itembox}{実行結果}
5
\end{itembox}

\subsection{2次元配列の初期化}
 2次元配列の初期化は1次元配列の初期化と少しだけ似ています。

\lstinputlisting{\codepath/doublearray2.c}

\begin{itembox}{実行結果}
6
\end{itembox}

\subsection{2次元配列の参照}
 2次元配列の配列の中身を確認したり、配列を扱うときは一緒にfor文を扱うことが多いです。

\lstinputlisting{\codepath/doublearray3.c}

\begin{itembox}{実行結果}
6 numbers put on please.\\
2 3 4 5 6 7\\
2 3 4 5 6 7
\end{itembox}
表示的には先ほどの1次元配列と変わりません。
ちなみに3次元以上の多次元配列をつくることもできます。
2次元配列とほぼ似たような宣言、参照をするため省略します。

\section{復習問題}


\end{document}
