\section{2次元配列}
 1次元配列はary[1]のように添え字が1つでした。2次元配列は単純にary[1][1]のように添え字が2つになった配列のことを言います。
ちなみにこれはオセロや五目並べの盤面、グリッドなどをつくるときによく使います。

\subsection{2次元配列の宣言}
 まずは2次元配列の宣言の仕方です。宣言方法は以下のようになっています。

\begin{itembox}{2次元配列の宣言方法}
型名 配列名[要素数][要素数];\\
例)int array[3][4];
\end{itembox}

ちなみに2次元配列にも次のように代入できます。
\lstinputlisting{\codepath/doublearray1.c}

\begin{itembox}{実行結果}
5
\end{itembox}

\subsection{2次元配列の初期化}
 2次元配列の初期化は1次元配列の初期化と少しだけ似ています。

\lstinputlisting{\codepath/doublearray2.c}

\begin{itembox}{実行結果}
6
\end{itembox}

\subsection{2次元配列の参照}
 2次元配列の配列の中身を確認したり、配列を扱うときは一緒にfor文を扱うことが多いです。

\lstinputlisting{\codepath/doublearray3.c}

\begin{itembox}{実行結果}
6 numbers put on please.\\
2 3 4 5 6 7\\
2 3 4 5 6 7
\end{itembox}
表示的には先ほどの1次元配列と変わりません。
ちなみに3次元以上の多次元配列をつくることもできます。
2次元配列とほぼ似たような宣言、参照をするため省略します。