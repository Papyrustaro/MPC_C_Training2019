\section{演習問題}
\subsection{問1}
\subsubsection{解答}
\lstinputlisting{\codepath/03_Answer1.c}
\subsubsection{解説}
for文により、ある回数だけ繰り返す処理を書く練習である。
もちろん、入力から受け取った整数Xの回数だけ繰り返すことも可能である。
ついでにあまりにも長い文章はコピペする手法も覚えよう。

%ここに解説

\subsection{問2}
\subsubsection{解答}
\lstinputlisting{\codepath/03_Answer2.c}
\subsubsection{解説}
for文のループカウンタを足していく手法で、総和を求めた。
何か別の変数を0で初期化して、iを全て足していくことで総和を求めることが出来る。
%ここに解説

\subsection{問3}
\subsubsection{解答}
\lstinputlisting{\codepath/03_Answer3.c}
\subsubsection{解説}
変数においてもfor文の初期条件や終了条件にできます。
i++も、i+=2とか、printf()とか、いろんなものに変えられるので試してみると面白いです。
実は、任意のX,Y(X<Y)について、高速に和を求める手法があるので興味があれば調べてみること(数学の知識が必要になります)。
%ここに解説

\subsection{問4}
\subsubsection{解答}
\lstinputlisting{\codepath/03_Answer4.c}
\subsubsection{解説}
for文の中で,if文を用いた制御をする問題である。
3の倍数でなければ足し算する...というイメージだが、様々な書き方があるので、自分に合った書き方を身につけましょう。
コメントアウトした方法でも求めることが出来る。
%ここに解説

\section{榊くんからの挑戦状}
\subsection{問1}
\subsubsection{解答}
\lstinputlisting{\codepath/03_AnswerS1.c}
\subsubsection{解説}
これはFizzBuzzと呼ばれる、非常に典型的な問題である。
インターネットで検索すれば、解説は山ほど書いてある。
この問題はelse ifを使わないで実装することも可能なので、興味があれば書いてみると勉強になるかもしれない。
%ここに解説


\subsection{問2}
\subsubsection{解答}
\lstinputlisting{\codepath/03_AnswerS2.c}
\subsubsection{解説}
九九の表は、縦をi,横をjとして1から9までの各i,jでi*jを出力する問題である。
二重for文の動作を確認するにはうってつけの問題である。
9と書かれている部分を変えてみたり、改行を消してみたりして、コードを改造しつつ理解の手助けにしてほしい。
%ここに解説


\subsection{問3}
\subsubsection{解答}
\lstinputlisting{\codepath/03_AnswerS3.c}
\subsubsection{解説}
出力する値が奇数ならば出力せずにその繰り返しを終え、偶数ならば出力するプログラムである。
2重for文もif文による制御で、様々なことを実現させることが出来る。
また、繰り返し文は何重になろうとも、if文、break文、continue文などの制御が出来る。
%ここに解説


\subsection{問4}
\subsubsection{解答}
\lstinputlisting{\codepath/03_AnswerS4.c}
\subsubsection{解説}
今回は,i<jの時に何も出力しないようにする問題である。
for文をif文によって操作するのは、非常によく使う手法なので、確実に覚えておいてほしい。
for文とcontinue,break文に慣れれば、問3,4は簡単に思えてくるはずだ。
極僅かにbreakの方が早い。
%ここに解説


\subsection{問5}
\subsubsection{解答}
\lstinputlisting{\codepath/03_AnswerS5.c}
\subsubsection{解説}
C言語の学習が進んでいる人は、配列を使えば良いと思うだろう。
実際、そのようにしたほうがこの問題は解きやすい。
2で割った余りで操作を変更するのは、偶数か奇数かを考えることである。
少し数学的な問題だったかもしれないが、コンピュータの内部では0と1のみでデータを表しており、情報の世界では2で割った余り(0,1)は非常によく出てくる。

余談だが、bit演算子を用いることでif(i\& 0)のようなコードに変更することができるはずである。
難しいので、解説ではそういうものもあるということだけ触れておく。
興味があれば「C言語 bit演算子」「論理演算」等で検索して調べてみましょう。
興味があってわからなければ榊まで問い合わせてください。
%ここに解説

