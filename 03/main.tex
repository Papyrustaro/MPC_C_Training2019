%------------------------------------- ページサイズなどの書式設定
%¥documentclass[a4j,twocolumn, dvipdfmx]{jsarticle} % 二段組の構成にする
%¥documentclass[a4j,notitlepage]{jsarticle} % タイトルだけのページを作らない
\documentclass[a4j,titlepage,dvipdfmx]{jsarticle}   % タイトルだけのページを作る
%------------------------------------- パッケージ読み込み
\newcommand{\stypath}{./sty}
\newcommand{\codepath}{./code/03}
\newcommand{\articlepath}{./article}
\usepackage[ipaex]{pxchfon}
%\usepackage{itembkbx}
\usepackage{\stypath/listings}
\usepackage{ascmac}
\usepackage{\stypath/jlisting}
\lstset{% 
showstringspaces=false,%空白文字削除
language={C},% %言語選択
basicstyle={\upshape},% %標準の書体
identifierstyle={\small},% %キーワードでない文字の書体
ndkeywordstyle={\small},% %キーワードその2の書体
stringstyle={\small\ttfamily},% %””で囲まれた文字などの書体
frame={tb},% %枠、デザインなど
breaklines=true,% %行が長くなった時の自動改行
columns=[l]{fullflexible},% %書体による列幅の違いを調整するか
numbers=left,% %行番号を表示するか
xrightmargin=0zw,% %余白の調整?
xleftmargin=0zw,% %余白の調整
numberstyle={\scriptsize},%行番号の書体
stepnumber=1,% %行番号をいくつ飛ばしで表示するか
numbersep=1zw,% %行番号と本文の間隔
morecomment=[l]{//}% 
} 
\title{C言語講座第三回・ループ文}
\author{MPC部員}
\date{2019年5月23日}
\begin{document}
\maketitle
\section{前回の復習}
\begin{enumerate}
\item if,if-else if-else if-else
\item switch-case
\end{enumerate}
\section{値の代入操作とインクリメント・デクリメント}
\subsection{値の代入操作}
第一回の時にC言語における=は左の変数に右の数式の計算結果を代入するという話をしました.\\
それの復習を行います.
\begin{lstlisting}
number=334;
\end{lstlisting}
この代入操作には何も違和感がありませんが次の場合はどうでしょう.
\begin{lstlisting}
number=number+1;
\end{lstlisting}
この代入操作はおかしく見えますが正しい操作です.これはnumberに+1した数字をnumberに代入するという操作です.
この操作は少し省略して書くことができます.

\begin{lstlisting}
number+=1
\end{lstlisting}
この表記は先ほどと同じ操作です.
具体的には
\begin{itembox}{値の代入}
"変数" "計算操作"="変数を操作したい値"
\end{itembox}
という風になります.
少し見辛いですがよく使う表現ですのでよく確認しておきましょう

\subsection{インクリメント・デクリメント}
%ここからはまだ
\subsection{前置と後置}
%\section{繰り返し操作}
\section{ループ文}
\subsection{for文}
 コンピュータにい待った回数同じ操作を行ってほしいというときにfor文を使います.
まずは下のプログラムを実行してみましょう.


\lstinputlisting{\codepath/for.c}

\begin{itembox}{実行結果}
0 1 2 3 4 5 6 7 8 9

\end{itembox}
このプログラムを見てわかるように,i++のおかげでiが1回ループするごとに1増加していることがわかります.
また,for文の基本的な内容はこのように泣ており,条件式を満たすまでループを行うという仕組みです.
変化式は値の変化を表すi++やi--が入ります.

\begin{itembox}{for文の書き方}
for(初期値;条件式;値の変化式)\{ \\
~~~~実行したい処理など;\\
\}
\end{itembox}
\subsection{while文}

\subsection{do-while文}

%\section{制御}
\section{制御文}
\subsection{break文}

\subsection{continue文}

\subsection{ネスト構造}

\section{演習問題}
勝手に文章化した榊くんの問題評価(読まないで良いです)\\
全ての繰り返し文は、for文でもwhile文でもdo-while文でも書くことが出来る。\\
一番使用頻度が多いのはfor文だと思われる。\\
問1から問3までが繰り返し文の動作確認\\
問4は繰り返し文の中で条件分岐をする練習\\
榊くんの挑戦状は、難易度BASICから手ごわい問題ばかりだが、解ければ本講座の内容は完全に理解したといえるだろう。\\
EX問題は晩御飯奢られチャレンジをしたい人のみ挑戦してみてください。今回のEX問題は一応、今まで習ったものを組み合わせれば解けます。\\

\subsection{問1}
出題意図:for文によりある回数だけ同じことを繰り返せることを理解する。\\

Steel is my body, and fire is my blood.I have created over a thousand blades. Unknown to Death. Nor known to Life.Have withstood pain to create many weapons.Yet, those hands will never hold anything.So as I pray, unlimited blade works.\\
を10回出力せよ。1回毎に改行すること。なお、これはFateが元ネタである。\\
こんなものをいちいちタイピングしては大変なので、コピーアンドペーストをすること。(コピペ)\\

\subsection{問2}
出題意図:for文のループカウンタを使う練習をする。\\
\\
1から200までの総和を求めるプログラムを作成せよ。数学で習った公式は使わずに実装しなさい。\\
即ち、for文を用いて1から200までを全て足し算しなさい。\\

\subsection{問3}
出題意図:for文の初期条件、終了条件を大まかに理解する。\\
\\
問2のコードを改造し、XからY(例えば20から183)までの総和を求めるプログラムにしなさい。\\
ただし、X,Yは標準入力から受け取ること(つまりscanfで読み取ること)。\\
入力例:\\
20 183\\
出力例:\\
16646\\

\subsection{問4}
\begin{verbatim}
出題意図:for文内でif文を使って制御する方法を身につける。

問3のコードを改造し、3の倍数の時には加算しないようにせよ。
第1回のテキストに記載されているが、判定したい数xに対してx%3=0が成り立てば3の倍数である。
入力例:
20 183
出力例:
11036
\end{verbatim}

\section{榊くんの挑戦状}

\subsection{問1 BASIC}
\begin{verbatim}
出題意図:典型問題「FizzBuzz」から,for文とif,if else文を理解する。

for文を用いて1からNまでを出力せよ。
ただし3の倍数の時は「Fizz」を,5の倍数の時は「Buzz」を,3の倍数かつ5の倍数の時は「FizzBuzz」を変わりに出力しなさい。
入力例:
16
出力例:
1
2
Fizz
4
Buzz
Fizz
7
8
Fizz
Buzz
11
Fizz
13
14
FizzBuzz
16
\end{verbatim}

\subsection{問2 BASIC}
\begin{verbatim}
出題意図:2重for文とその動作を理解する。

九九の表を出力せよ。例えば、printfのフォーマット指定子%dを%3dとすると出力がきれい並ぶようになる。
表らしくするため、各段の出力毎に改行すること。
  1  2  3  4  5  6  7  8  9
  2  4  6  8 10 12 14 16 18
  3  6  9 12 15 18 21 24 27
  4  8 12 16 20 24 28 32 36
  5 10 15 20 25 30 35 40 45
  6 12 18 24 30 36 42 48 54
  7 14 21 28 35 42 49 56 63
  8 16 24 32 40 48 56 64 72
  9 18 27 36 45 54 63 72 81
このような感じで出力されれば正解である。多少出力の形式が異なっていても良い。
\end{verbatim}

\subsection{問3 BASIC}
\begin{verbatim}
出題意図:continue文,if文による操作ができるか。for文の2重ループを適用できるか。

前問で、九九の表を作った。この表のうち奇数となる出力をしないようにせよ。
いびつな表になるが、それでよいのだ。
次の問題でも九九の表を使うため、前問のコードは	どこかにコピーしておくこと。
  2  4  6  8
  2  4  6  8 10 12 14 16 18
  6 12 18 24
  4  8 12 16 20 24 28 32 36
 10 20 30 40
  6 12 18 24 30 36 42 48 54
 14 28 42 56
  8 16 24 32 40 48 56 64 72
 18 36 54 72
このように出力されれば正解である。以下のようにしても良い。
     2     4     6     8
  2  4  6  8 10 12 14 16 18
     6    12    18    24
  4  8 12 16 20 24 28 32 36
    10    20    30    40
  6 12 18 24 30 36 42 48 54
    14    28    42    56
  8 16 24 32 40 48 56 64 72
    18    36    54    72
\end{verbatim}

\subsection{問4 BASIC}
\begin{verbatim}
出題意図:break文(continue文),if文による操作ができるか。for文の2重ループを適用できるか。

2つ前の問題で、九九の表を作った。この表のうち縦方向の値をi,横方向の値をjとしたとき、i<jを満たす出力をしないようにせよ。
  1
  2  4
  3  6  9
  4  8 12 16
  5 10 15 20 25
  6 12 18 24 30 36
  7 14 21 28 35 42 49
  8 16 24 32 40 48 56 64
  9 18 27 36 45 54 63 72 81
このように出力されれば正解である。
\end{verbatim}

\subsection{問5 BASIC}
\begin{verbatim}
出題意図:問題文の求める意味を理解してコードを書こう。

int型の変数aを宣言し、20回だけ標準入力から整数を受け取る。
この1回目、3回目、5回目...つまり奇数回目に読み取った整数のみを出力しなさい。
例を挙げる。
1 64 2 7 5 75 334 84 32 7 43 74 24 21 12 55 184 24 376 256
が与えられた場合は,
1 2 5 334 32 43 24 12 184 376
と出力すること。空白区切りが難しければ、
1
2
5
334
32
43
24
12
184
376
として出力毎に改行しても良い。
\end{verbatim}

\section{榊くんの挑戦状EX}
この大問は、演習問題や榊くんの挑戦状が簡単すぎた人へ贈る鬼畜問題である。
これが解けなくても全く問題はないので安心してほしい。
具体的には、学部2年生でも1割の学生が解けるかな...?というような難易度である。
今回のEX問題は2つの小課題に分類される。EX問1でlong intについて調べれば、どちらも今までの講座の文法で解くことが出来る。
演習時間内にEX課題問1かEX課題問2に正解すれば500円以内の品物やジュースを,両方を正解した人には晩御飯を奢ります。
(新学部1年先着1名限定)

\subsection{EX問1}
\begin{verbatim}
フィボナッチ数列というものがある。その定義は以下のようなものである。
f[0]=0,f[1]=1,
f[n]=f[n-1]+f[n-2](n>=2)
具体的には 0 1 1 2 3 5 8 13 21 ...と続く数列である。
フィボナッチ数列の第0項から第80項までを順に出力せよ。
ただし、配列を使用してはいけない。
*int型ではオーバーフローする点に注意すること。
*フィボナッチ数列第80項は23416728348467685という非常に大きな値なので、int型の最大値を超えているのだ。
上記の注意の意味が分からなければ、「C言語 long int」などと検索して使い方を調べること。
\end{verbatim}

\subsection{EX問2}
\begin{verbatim}
榊くんは今、3種類の硬貨(50円,100円,500円)をそれぞれ、0~7000個だけ持っているらしいです。
榊くんは今、所持金額の合計がX円であると言い張っています。
榊くんが今持っている硬貨の組み合わせとしてあり得るものの総数を求めなさい。
ただし、解を導くのに2秒以上かかる提出だった場合は途中点とし、榊くんは解答者に対してジュース1本を奢るものとする。
例を2つ示す。入力は榊くんの今の所持金を表す1つの数である。

入力例1:
250
出力例1:
3

1.50円が5枚と100円が0枚,500円が0枚
2.50円が3枚と100円が1枚,500円が0枚
3.50円が1枚と100円が2枚,500円が0枚
という3つの組み合わせがあり得るので、3と出力する。

入力例2:
1874
出力例2:
0

この場合は、3種類の硬貨をどのように組み合わせても、あり得る組み合わせは考えられないので0と出力する。
(榊くんはうそつきだ!!)
\end{verbatim}
\end{document}
