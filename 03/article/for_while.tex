\section{ループ文}
\subsection{for文}
 コンピュータに決まった回数同じ操作を行ってほしいというときにfor文を使います.
まずは下のプログラムを実行してみましょう.


\lstinputlisting{\codepath/for.c}

\begin{itembox}{実行結果}
0 1 2 3 4 5 6 7 8 9

\end{itembox}
このプログラムを見てわかるように,i++のおかげでiが1回ループするごとに1増加していることがわかります.
また,for文の基本的な内容はこのようになっており,条件式を満たすまでループを行うという仕組みです.
変化式は値の変化を表すi++やi\verb"--"が入ります.

\begin{itembox}{for文の書き方}
\begin{verbatimtab}
for(初期値;条件式;値の変化式){ 
	//実行したい処理など;
}
\end{verbatimtab}
\end{itembox}
\subsection{while文}
while文はコンピューターに不特定回数の同じ操作をさせるときに使います.\\
不特定回数なのであればどのようにループさせるかと言うと条件を指定して,その条件を満たしている間はループします.\\
次のプログラムを実行してみましょう.\\
\lstinputlisting{\codepath/while.c}
このプログラムはfor文の項目で書いたものと同じ操作を行います.\\
現時点だとそこまで有用ではないですが次回の講習を行った後にはすごく使えるものとなりますので覚えておきましょう.\\
\begin{itembox}{while文の書き方}
\begin{verbatimtab}
while(条件式){
		//実行したい処理
}
\end{verbatimtab}
\end{itembox}
\subsection{do-while文}
まず次のプログラムを実行してください
\lstinputlisting{\codepath/while2.c}
実行しても何も起こりませんね.\\
このプログラムは本来1か2を入力されるまでエラーを表示するプログラムのはずなんですが,最初にi=0;としているせいでうまく動きません.それはループに入るときに一番最初に条件をチェックしてループしているからです.\\
このプログラムをdo-while文を使って書き直してみましょう.\\
\lstinputlisting{\codepath/do_while.c}
do-while文はループ内部の処理を行った後に条件判定をしてもう一度ループするか,しないかを決めています.\\
なのでこのように正規の入力を待つ場合はdo-while文がオススメです.\\
\begin{itembox}{do-while文の書き方}
\begin{verbatimtab}
do{
	//ループしたい処理
}while(条件式);
\end{verbatimtab}
\end{itembox}
