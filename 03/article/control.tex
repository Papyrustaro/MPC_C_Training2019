\section{制御文}
\subsection{break文}
break文はループ処理から強制的に脱出する文です.\\
また前回行ったswitch文でも使います.\\
下のプログラムを実行してみましょう.\\
\lstinputlisting{\codepath/break.c}
\begin{itembox}[l]{実行結果}
\begin{verbatim}
こんにちは 1
こんにちは 2
こんにちは 3
こんにちは 4
こんにちは 5
\end{verbatim}
\end{itembox}

このように条件を書くことができますが,if文でbreakをすることはループ文の条件をみづらくするので推奨されません.\\
なので今のプログラムなら\\
\lstinputlisting{\codepath/nobreak.c}
このように書くべきですね.\\
\subsection{continue文}
現在のループ処理を中断して次のループ処理に移動するcontinue文というものもあります.\\
以下のプログラムを実行してみてください.\\
\lstinputlisting{\codepath/continue.c}
これも先ほどのbreak文と同様にループ文の条件をみづらくするので推奨されません.気をつけましょう.\\
\subsection{ネスト構造}
if文などと同様にfor文やwhile文も入れ子構造にすることができます.\\
以下のプログラムを実行してみましょう.\\
\lstinputlisting{\codepath/Wfor.c}
この方法を使うと様々なループをかけますがあまり入れ子構造にしすぎると可読性(読みやすさ)が下がりますので気をつけましょう.\\

