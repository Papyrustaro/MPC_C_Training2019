%------------------------------------- ページサイズなどの書式設定
%¥documentclass[a4j,twocolumn, dvipdfmx]{jsarticle} % 二段組の構成にする
%¥documentclass[a4j,notitlepage]{jsarticle} % タイトルだけのページを作らない
\documentclass[a4j,titlepage,dvipdfmx]{jsarticle}   % タイトルだけのページを作る
%------------------------------------- パッケージ読み込み
\newcommand{\stypath}{./sty}
\newcommand{\codepath}{./code}

\usepackage[ipaex]{pxchfon}
%\usepackage{itembkbx}
\usepackage{\stypath/listings}
\usepackage{ascmac}
\usepackage{\stypath/jlisting}
\lstset{% 
showstringspaces=false,%空白文字削除
language={C},% %言語選択
basicstyle={\upshape},% %標準の書体
identifierstyle={\small},% %キーワードでない文字の書体
ndkeywordstyle={\small},% %キーワードその2の書体
stringstyle={\small\ttfamily},% %””で囲まれた文字などの書体
frame={tb},% %枠、デザインなど
breaklines=true,% %行が長くなった時の自動改行
columns=[l]{fullflexible},% %書体による列幅の違いを調整するか
numbers=left,% %行番号を表示するか
xrightmargin=0zw,% %余白の調整?
xleftmargin=0zw,% %余白の調整
numberstyle={\scriptsize},%行番号の書体
stepnumber=1,% %行番号をいくつ飛ばしで表示するか
numbersep=1zw,% %行番号と本文の間隔
morecomment=[l]{//}% 
} 
\title{C言語講座第2回演習問題回答}
\author{那由多・榊}
\date{2019年5月16日}

\begin{document}

\section{榊くんの挑戦状}
EX問題の解答を知りたい人は、榊まで連絡をください。
\subsection{問1}
\subsubsection{模範解答}
\lstinputlisting{\codepath/sakaki_exe01.c}
\subsubsection{解説}
%ここに解説
if文,else if文を正しく適用できるかを問う問題であった。\\
特に気を付けなければいけないのは、a=bの時であろう。\\
C言語の=は代入の意味で使われるので、\\
if(a=b){;}などというコードを書いてしまっては、正しく条件分岐が出来なくなってしまう。\\
さらに、コンパイルは通るし、実行は可能であるので、予想外のバグにもなり得る。\\
動作が気になる人は、if(a=b){;}の動作も確認してみると良い。\\

\subsection{問2}
\subsubsection{模範解答}
\lstinputlisting{\codepath/sakaki_exe02.c}
\subsubsection{解説}
%ここに解説
\begin{verbatim}
if文と、else if文における複数条件、もしくはネスト構造によって書くことが出来るかを問う問題であった。
論理演算では&&と||に優先順位があり、不安ならば優先させたい項目を()でくくることで、間違いなくコーディングすることができる。
\end{verbatim}
\subsection{問3}
\subsubsection{模範解答}
\lstinputlisting{\codepath/sakaki_exe03.c}
\subsubsection{解説}
%ここに解説
\begin{verbatim}
問題文の意図を正しく読み取って、必要なif文を自分で作成する問題であった。
a,b,c,dの中の最大値を求めるには、まず、暫定の最大値max_numberとしてaを代入し、
bがそれを超えていればbに変更、cがそれを超えていれば......とすることで、全ての最大値を求めることが出来る。
同様にすることで、最小値も求められる。
他の解答としては、多重のネストを構成することで、全パターンの最大値をif文に書き表す方法もある。
\end{verbatim}
\subsection{問4}
\subsubsection{模範解答}
\lstinputlisting{\codepath/sakaki_exe04.c}
\subsubsection{解説}
%ここに解説
if文はただ変数同士を比較するだけでなく、a+b等の四則演算の結果同士も比較することが出来る。\\
これが問題なく理解できれば、後は他の問題と同じである。\\
\end{document}
